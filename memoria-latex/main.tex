\documentclass[12pt]{article}
\usepackage[margin=2.54cm]{geometry}

\usepackage{graphicx}
\usepackage[utf8]{inputenc}
\usepackage{csquotes}

\graphicspath{ {./imgs/} }

\usepackage[spanish]{babel}
\usepackage{hyperref}
\hypersetup{
  colorlinks=true,  % color en lugar de recuadros
  linkcolor=black,  % enlaces internos
  urlcolor=blue   % enlaces externos
}
\usepackage{url}
\usepackage{float}
\usepackage{enumitem}
\usepackage{comment}
\usepackage{wasysym}
\usepackage{multirow}
\usepackage[utf8]{inputenc}
\usepackage[usenames]{color}
\usepackage[document]{ragged2e}
\usepackage[table]{xcolor}
\usepackage{colortbl}
\definecolor{lightgray}{rgb}{0.9, 0.9, 0.9}
\definecolor{black}{rgb}{0, 0, 0}
\renewcommand{\arraystretch}{1.3}
\arrayrulecolor{black}
\setlength{\arrayrulewidth}{0.8pt}

\usepackage[backend=biber,style=apa]{biblatex} % Citación APA
\addbibresource{bibliografia.bib}
\usepackage{csquotes}
\usepackage{setspace}
\onehalfspacing % Iinterlineado a 1.5
\usepackage{titlesec}

% Configuración de la fuente
\usepackage[T1]{fontenc}
\usepackage{newtxtext}

\begin{document}
\begin{titlepage}
    \centering
    \begin{minipage}{1\textwidth}
        \raisebox{-0.7\height}
        {\includegraphics[width=0.5\textwidth]{UGR-Logo}}
        \raisebox{-0.8\height}{\includegraphics[width=0.49\textwidth]{ETSIIT-logo.png}}
    \end{minipage}
   
    \vspace{1.5cm}

    {\LARGE Universidad de Granada \par}
    \vspace{0.5cm}

    {\Large Escuela Técnica Superior de Ingenierías Informática y de Telecomunicación \par}

    \vspace{1cm}
    
    {\LARGE {Trabajo de Fin de Grado (TFG)} \par}
    \vspace{1.5cm}

    {\Huge \textbf{Juego de la Oca en Alexa}:
    	
	iniciativa para fomentar el envejecimiento saludable de las personas mayores \par}
    
    \vspace{1.5cm}

    \vfill

    {\Large \textbf{Autora} \par}
    {\Large Marina Jun Carranza Sánchez \par}
    \vspace{0.5cm}

    {\Large \textbf{Tutora} \par}
    {\Large Nuria Medina Medina \par}
    
\end{titlepage}

\newpage
\justifying

\vline

\vspace{4cm}

\textbf{Agradecimientos:}

\vspace{1cm}

Quiero agradecer a todas las personas que me han acompañado durante este proyecto. A mis padres y amigos, por su apoyo incondicional, y a mi tutora, por su ayuda y asesoramiento.

\newpage

\textbf{Resumen:}

Este TFG desarrolla un juego digital interactivo con un dispositivo Alexa, dirigido a personas mayores que pueden vivir en Residencias o asistir a Centros de Día, basado en el tradicional juego de la oca que incluye minijuegos que le aportan originalidad y mucho entretenimiento, todo ello con el objetivo de contribuir al desarrollo cognitivo y emocional así como a la interacción social. 

El trabajo está inspirado en el proyecto de investigación <<Evaluación del uso de robots sociales y sistemas conversacionales en Residencias y Centros de Día para promover el envejecimiento saludable>> de la Universidad de Granada. Alexa es el asistente virtual escogido para el juego a desarrollar, debido a que sus skills ofrecen una amplia gama de posibilidades para crear experiencias interactivas y entretenidas. Estas habilidades permiten a los desarrolladores crear juegos de diferentes géneros y niveles de complejidad, desde simples juegos de palabras y adivinanzas hasta juegos de aventuras o trivial más elaborados.

\vline

\textbf{Abstract:}

This Bachelor's Thesis develops an interactive digital game with an Alexa device, aimed at elderly people who may live in nursing homes or attend day centers. It is based on the traditional game of Goose and includes mini-games that add originality and much entertainment, all with the goal of contributing to cognitive and emotional development as well as social interaction.

The work is inspired by the research project <<Assessment of the Use of Social Robots and Conversational Systems in Nursing Homes and Day Centers to Promote Healthy Aging>> at the University of Granada. Alexa is the virtual assistant chosen for the game being developed, due to its skills offering a wide range of possibilities for creating interactive and entertaining experiences. These skills allow developers to create games of different genres and levels of complexity, from simple word games and riddles to more elaborate adventure or trivia games.

\vline

\textbf{Palabras clave:}

Gamificación, aislamiento, personas mayores, asistentes de voz, asistentes conversacionales, Alexa, skill, envejecimiento activo, envejecimiento saludable, juegos serios, juegos educativos, residencias, centros de día.

\vline

\textbf{Key words:}

Gamification, isolation, elder people, voice assistants, chatbots, Alexa, skill, active aging, healthy aging, serious games, educational games, nursing homes, adult daycare centers

\newpage
\tableofcontents

\newpage
\listoffigures

\newpage
\listoftables

\newpage

\section{Introducción}
Este Trabajo de Fin de Grado se enfoca en el desarrollo de un juego para asistentes conversacionales, dirigido a adultos mayores en residencias o centros de día, que pueden ser propensos a aislarse a pesar de estar rodeadas de otras personas. Reconociendo los desafíos que enfrenta este grupo demográfico en la era digital, el proyecto busca integrar una de las tecnologías más punteras de manera accesible y lúdica. Por tanto, el objetivo es no solo fomentar la conexión social en estos espacios, sino también promover la estimulación cognitiva y el bienestar emocional de las personas mayores.

Dado que las personas mayores están menos familiarizadas con el uso de herramientas digitales en su vida cotidiana puede parecer difícil que se sientan atraídas por juegos que se basan en la tecnología, sin embargo si las personas usuarias en un primer momento comprueban que no solo pueden acceder a este entretenimiento si no que, además se incentivan sus relaciones personales, conocen nuevas amistades y se abre un mundo nuevo de ocio y desarrollo intelectual, se podrá ver cómo conduce a un envejecimiento más saludable.

La idea detrás de este TFG podría resumirse con la siguiente cita: \enquote{Las actividades físicas, cognitivas y emocionales en edades avanzadas son cruciales para estimular la actividad cerebral y contribuir al mantenimiento de la calidad de vida. Además de los beneficios físicos y cerebrales, los juegos estimulan la interacción social y contribuyen a la socialización y al mantenimiento de la salud emocional y afectiva} \parencite{intro3}.


\subsection{Motivación y contexto}

Este TFG se integra dentro del proyecto de investigación \enquote{Evaluación del uso de robots sociales y sistemas conversacionales en residencias y centros de día para promover el envejecimiento saludable} de la Universidad de Granada, con  la profesora Nuria Medina Medina como investigadora principal del mismo, y directora de este Trabajo de Fin de Grado.

Este proyecto, con código C-ING-179-UGR23, \enquote{propone el uso y evaluación de Agentes Sociales Interactivos (SIA - Social Interactive Agent) (en particular de Robots Sociales apoyados por Asistentes Conversacionales) que promuevan el envejecimiento saludable y las interacciones sociales de los mayores en la residencia o centro de día, ya que estas interacciones son muy importantes para el mejoramiento de la salud y estado anímico de los mayores. Para maximizar su efectividad, la propuesta integrará experiencias lúdicas y técnicas de interacción multimodal}. 

Dicho proyecto destaca por su metodología pionera en relación con el estado del arte ya que se focaliza en la utilización de los robots sociales para fomentar el envejecimiento saludable y mejorar la comunicación y los problemas de interacción social en las residencias. El proyecto también se centra en el diseño de experiencias lúdicas integradas en entornos sociales para favorecer los problemas de motivación y adopción de la tecnología.

Que el uso de la tecnología puede ser un apoyo importante para lograr una mayor calidad de vida en los adultos mayores que viven en residencias o asisten a centros de día es una de las premisas en las que se apoya la investigación. \enquote{Consecuentemente, este proyecto propone el uso y evaluación de Agentes Sociales Interactivos (SIA - Social Interactive Agent) (en particular de Robots Sociales apoyados por Asistentes Conversacionales) que promuevan el envejecimiento saludable y las interacciones sociales de los mayores en la residencia o centro de día, ya que estas interacciones son muy importantes para el mejoramiento de la salud y estado anímico de los mayores. Para maximizar su efectividad, la propuesta integrará experiencias lúdicas y técnicas de interacción multimodal}.

Según se menciona en el artículo \textit{La soledad y el aislamiento social en las personas mayores} \parencite{ArruebarrenaCabaco2020}, \enquote{El aislamiento social se define como una ausencia objetiva de relaciones/contactos sociales y la soledad como la experiencia subjetiva aversiva que se siente al valorar esas relaciones/contactos sociales como
insuficiente en cantidad y/o calidad}

En los últimos tiempos, el tema de la soledad en las personas mayores ha ganado atención en los medios de comunicación, describiéndose como una \enquote{epidemia} en aumento. Aunque no hay evidencia sólida que respalde la idea de una nueva epidemia, las dificultades metodológicas y la falta de consenso en la medición de la soledad limitan la capacidad de confirmar si las personas mayores se sienten más solas que antes.

A pesar de estas limitaciones, estudios indican tasas de soledad entre las personas mayores en España, oscilando entre el 14\% y el 24\%, e incluso alcanzando el 40\% en algunos casos. Uno de los factores que contribuyen a esta percepción es el aumento en el número de personas mayores viviendo solas. Se proyecta que para 2050, aproximadamente un tercio de la población tendrá más de 65 años, lo que por lógica implica un aumento en el número de personas mayores que viven solas.

\begin{figure}[ht]
    \centering
    \includegraphics[width=0.98\textwidth]{imgs/piramide-poblacion.jpg}
    \caption{Pirámides de población de España en futuros años (\href{https://www.geriatricarea.com/2020/09/25/uno-de-cada-tres-espanoles-tendra-65-o-mas-anos-en-el-2050/}{geriatricarea.com})}
    \label{fig:piramide-poblacion}
\end{figure}

El aumento de la esperanza de vida de la población adulta en nuestro país ha supuesto que las personas mayores, muchas de ellas que viven solas y poseen nivel económico y cultural medio, supongan un sector amplio de la población que requiere de nuevas formas de ocio y de relacionarse socialmente. Muchas de estas personas viven solas y el sedentarismo y la falta de interacción social les produce que su desarrollo cognitivo se vea mermado.

También la frecuente automarginación de este grupo demográfico para el uso de herramientas digitales como juegos o aplicaciones, el fenómeno conocido de forma general como brecha digital, contribuye a un mayor aislamiento social. \enquote{Muchos adultos mayores tienen acceso a dispositivos móviles, pero no pueden aprovecharlos completamente debido a la falta de conocimiento o el miedo a salir de su zona de confort. Esto resulta en barreras emocionales, dificultades para adquirir nuevas habilidades tecnológicas} \parencite{intro1}. 

A pesar de lo mencionado en el párrafo anterior, el segmento de edad mayor de 60 años no es ajeno a la realidad de que las formas de socialización del siglo XXI están vinculadas a los avances tecnológicos y a las nuevas experiencias lúdicas. Como se señala en \textit{Las competencias digitales en personas mayores: de amenaza a oportunidad}: \enquote{el potencial que para las personas mayores ofrece el uso habitual de las TIC es enorme, con una larga lista de oportunidades existentes para el beneficio de este colectivo que deben ser aprovechadas} \parencite{intro4}.

Un juego digital que suponga entretenimiento para las personas mayores, al mismo tiempo que una mejora de memoria y ampliación de lenguaje y percepción puede significar un estimulante cambio en su día a día. 

\subsubsection{Impacto social y tecnológico de la pandemia de COVID-19}

Durante la pandemia por el COVID-19 millones de personas en el mundo tuvieron que pasar en pocos días de trabajo presencial al online y en sus relaciones personales debieron adaptar sus costumbres al nuevo escenario virtual.

De esta manera, personas acostumbradas a comunicarse solo por llamadas telefónicas, y muy poco mediante telefonía móvil, se volvieron usuarias de videollamadas diarias ante la necesidad de mantenerse conectados con familiares y amigos, combinada con las restricciones de movimiento.

Este cambio tuvo un impacto profundo en la vida diaria de las personas mayores. Por un lado, les brindó una forma vital de mantenerse conectados con sus seres queridos, incluso cuando no podían reunirse físicamente debido a las medidas de distanciamiento social. Esto ayudó a reducir un poco el riesgo de aislamiento social y proporcionó un medio para el apoyo emocional y la interacción social, lo cual es esencial para su bienestar mental y emocional.

Sin embargo, la transición a las tecnologías digitales también presentó desafíos, especialmente para aquellos menos familiarizados con ellas. Algunos enfrentaron dificultades técnicas al principio, como la configuración de aplicaciones o la resolución de problemas de conexión. Además, la dependencia excesiva de la tecnología para la comunicación también puede aumentar la brecha digital entre aquellos que tienen acceso y conocimientos tecnológicos y aquellos que no los tienen, lo que potencialmente podría aumentar el riesgo de exclusión social para algunos adultos mayores.

A pesar de estos desafíos, la pandemia actuó como un catalizador para la adopción de tecnología entre las personas mayores, lo que les permitió permanecer conectados y participar en la sociedad de manera más activa, incluso en tiempos de crisis. Como resultado, muchas personas mayores han incorporado el uso de tecnologías digitales en su vida diaria incluso después de que levantaran las restricciones de la pandemia, lo que les brinda nuevas oportunidades de participación social y acceso a recursos y servicios en línea \parencite{intro2}.

Según datos del Instituto Nacional de Estadística (INE), la población que usa Internet (en los últimos meses el uso de las tecnologías de información y comunicación (TIC) en los hogares ha crecido en los últimos años, si bien sigue existiendo una brecha entre los usuarios y no usuarios (brecha digital) que se puede atribuir a una serie de factores: la falta de infraestructura (en particular en las zonas rurales), la falta de conocimientos de informática y habilidades necesarias para participar en la sociedad de la información, o la falta de interés en lo que la sociedad de la información puede ofrecer).

Al aumentar la edad desciende el uso de Internet, siendo el porcentaje más bajo el que corresponde al grupo de edad de 65 a 74 años (un 79,7\% para los hombres y un 80,5\% para las mujeres) \parencite{intro5}.

\begin{figure}[H]
	\centering
	\includegraphics{imgs/INE-grafica1.jpeg}
	\caption{Población por grupos de edad que han usado Internet en los últimos tres meses (INE, 2018-23)}
	\label{fig:grafica1-INE}
\end{figure}

\begin{figure}[H]
	\centering
	\includegraphics{imgs/INE-grafica2.jpeg}
	\caption{Población de entre 16 y 74 años que ha usado Internet en los últimos tres meses en la UE (INE, 2020-23)}
	\label{fig:grafica2-INE}
\end{figure}

Según una encuesta realizada por Canal Sénior, plataforma online de entretenimiento y aprendizaje para personas mayores de 55 años, entre sus usuarios, en el capítulo de juegos y aplicaciones de entretenimiento: \enquote{la mayoría de los encuestados prefieren los juegos que permiten el entrenamiento mental, como aplicaciones del tipo Trivial, Apalabrados, Scrabble, Wordle, etc. Esto es muy relevante en personas sénior, pues está probado que los juegos que retan a la mente y nos hacen pensar pueden retrasar el envejecimiento cognitivo durante más tiempo}.

Todo hace indicar que cada vez hay una relación más estrecha entre mayores y ocio digital. \enquote{Es también relevante ver cómo algunos de sus gustos y preferencias son diferentes a los de otros grupos de edad, como por ejemplo su preferencia por la lectura al consumo de series y películas o de contenidos en las redes sociales. Por último, conviene destacar que el uso de las tecnologías digitales como parte de nuestro ocio pueden ayudarnos a tener un envejecimiento activo, sano y con participación elevada en la sociedad.} \parencite{intro6}.

\subsection{Objetivos}
El presente trabajo tiene como objetivo desarrollar un juego digital, que contribuya al creciente conjunto de soluciones innovadoras para combatir el aislamiento social en los adultos mayores alojados en residencias o centros de día.

Este juego no solo buscará proporcionar entretenimiento y diversión, sino que también se centrará en promover la participación, el compromiso cognitivo y emocional, y en general, mejorar la calidad de vida de las personas mayores. 

Para lograr este objetivo principal, se puede descomponer en varios subobjetivos, recogidos en la siguiente tabla:

\begin{table}[ht]
  \centering
  \begin{tabular}{| c | p{9.6cm} |}
    \hline
    \rowcolor{lightgray}
    \textbf{Subobjetivo} & \textbf{Descripción} \\
    \hline
    1. Revisión de la literatura & 
        1.1. Identificar las investigaciones clave sobre el impacto del aislamiento social en personas mayores \newline
        \vspace{0.2cm}
        1.2. Analizar la diversidad de intervenciones digitales dirigidas a este grupo demográfico \vspace{0.2cm} \\
    \hline
    2. Diseño del juego &
        2.1. Investigar las mejores prácticas en el diseño de juegos digitales accesibles para personas mayores \newline
        \vspace{0.2cm}
        2.2. Considerar las adaptaciones necesarias para abordar posibles limitaciones físicas y cognitivas
        \vspace{0.2cm} \\
    \hline
    3. Desarrollo técnico & 
        3.1. Seleccionar la plataforma y tecnologías más apropiadas para el desarrollo del juego \newline
        \vspace{0.2cm}
        3.2. Asegurar la compatibilidad con dispositivos comunes utilizados por personas mayores \newline
        \vspace{0.2cm}
        3.3. Integrar funcionalidades de accesibilidad, como ajustes de tamaño de fuente y navegación simplificada \vspace{0.2cm} \\
    \hline
    4. Contribución al conocimiento & 
    4.1. Contribuir al creciente cuerpo de conocimientos sobre cómo la tecnología puede mejorar la calidad de vida de los adultos mayores, particularmente en el contexto de residencias y centros de día. 
    \vspace{0.2cm} \\
    \hline
    5. Participación comunitaria & 
        5.1. Colaborar con comunidades de personas mayores para obtener retroalimentación. \newline
        \vspace{0.2cm}
        5.2. Organizar pruebas piloto y grupos de enfoque para evaluar la experiencia del usuario. \vspace{0.2cm} \\
    \hline
    \end{tabular}
  \caption{Subobjetivos del trabajo.}
  \label{tab:subobjetivos}
\end{table}

\newpage
\section{Estado del arte}

En la intersección entre el avance tecnológico y el envejecimiento de la población, los juegos digitales emergen como una herramienta multifacética con el potencial de promover el bienestar y la calidad de vida de los adultos mayores. El estado del arte representa una síntesis dinámica de la investigación, desarrollo y aplicaciones existentes en este campo en constante evolución.

En esta revisión, se explorarán las principales tendencias y avances en cuatro áreas clave: los juegos serios, diseñados con objetivos específicos de aprendizaje o rehabilitación; los juegos digitales, adaptados para satisfacer las necesidades y preferencias de las personas mayores; el estado actual de las tecnologías aplicadas, incluyendo dispositivos y software especializado en el diseño y finalmente, ejemplos de aplicaciones prácticas desarrolladas en esta línea de investigación.

\subsection{Juegos serios}

Los juegos \textit{serios} se distinguen por su enfoque principal en objetivos educativos o formativos, relegando la diversión a un segundo plano \parencite{juegosSerios}.
Mientras que los juegos tradicionales buscan principalmente entretener al jugador, los juegos serios utilizan la pedagogía para integrar la instrucción dentro de la experiencia de juego. Esto implica que, aunque la diversión sigue siendo un elemento presente, el aprendizaje se convierte en el objetivo primordial.

\subsubsection{Concepto de juego serio}
Los juegos serios son una categoría de videojuegos desarrollados principalmente para propósitos educativos y formativos, aunque también pueden tener elementos de entretenimiento. Estos juegos se distinguen por su enfoque en la transmisión de conocimientos y habilidades, a menudo relacionados con temas como la política, la salud, el entrenamiento militar, la educación, el ámbito empresarial, etc  \parencite{juegosSerios3}.

Se listan a continuación algunas de las características principales que distinguen a los juegos serios del resto:

\begin{itemize}[leftmargin=1.5cm, topsep=0pt, itemsep=1pt, after=\vspace{0pt}]
    \item \textbf{Fin educativo}: su principal objetivo es la formación en lugar del entretenimiento. La adquisición de habilidades técnicas y comprensión de procesos complejos.
    \item \textbf{Realismo}: están vinculados a aspectos de la realidad, favoreciendo la identificación del jugador con el contexto que se está simulando.
    \item \textbf{Ambiente virtual seguro}: proporcionan un entorno tridimensional virtual seguro para la práctica en ciertas áreas, como el entrenamiento militar.
    \item \textbf{Interés temático}: pueden abordar temas políticos, económicos, psicológicos, religiosos, entre otros, a menudo con un enfoque en la educación y el entrenamiento.
\end{itemize}

Los siguientes son algunos ejemplos reales y de ámbito específico de videojuegos serios:
\begin{itemize}[leftmargin=1.5cm, topsep=2.2pt, itemsep=1pt]
    \item \textbf{Biomedical Training}: entrenamiento para trabajadores de la salud en emergencias.
    \item \textbf{Food Force}: educación sobre el Programa Mundial de Alimentos de la ONU.
    \item \textbf{Incident Commander}: dirección de acciones en crisis sociales y desastres naturales.
    \item \textbf{Hazmat: Hotzone}: simulación para bomberos en situaciones de emergencia.
    \item \textbf{Yourselfittness}: regímenes de ejercicios físicos y aeróbicos.
    \item \textbf{Real Life 2007}: simulación global para entender condiciones de vida en diferentes países.
\end{itemize}

Los juegos serios son una herramienta poderosa para el aprendizaje y la formación en distintos campos. Involucran principios pedagógicos y cognitivos para garantizar un entrenamiento efectivo y ofrecen diversas ventajas como la familiarización con interfaces tecnológicas y la mejora de habilidades necesarias para la sociedad digital.

En el contexto de los adultos mayores, los videojuegos serios pueden resultar valiosos para el mantenimiento cognitivo, la estimulación mental y la mejora de habilidades específicas.

\subsubsection{Juegos serios en la rehabilitación neuropsicológica}

Cabe destacar el papel de los juegos serios en el ámbito de la informática médica, en concreto, en el entrenamiento y la rehabilitación neuropsicológica.  \enquote{En la rehabilitación cognitiva los retos de un juego serio por lo general inciden directamente en un déficit específico, lo que puede repercutir al mismo tiempo en más de uno.} \parencite{juegosSerios2} 

En los últimos años, ha habido un notable aumento en la investigación y desarrollo de juegos computarizados para la rehabilitación cognitiva. Ejemplos como RehaCom® y Gradior® \parencite{ICTestudios} han demostrado resultados positivos en varios centros de salud al ofrecer una variedad de ejercicios digitales que abordan áreas cognitivas clave como la atención, la percepción, la memoria y el lenguaje. 

\begin{figure}[ht]
    \centering
    \includegraphics[width=0.8\textwidth]{imgs/gradior.jpg}
    \caption{Gradior, un sistema de evaluación y rehabilitación neuropsicológica}
    \label{fig:gradior}
\end{figure}

Otro destacado ejemplo es el módulo EINK de RehaCom®, diseñado para tratar déficits de memoria de trabajo y planificación mediante la simulación de situaciones diarias, como ir de compras. Estos programas permiten adaptar la intervención terapéutica a las necesidades de cada paciente, ajustando la dificultad y la retroalimentación sobre su progreso.

En resumen, los juegos serios ofrecen una vía innovadora y efectiva para abordar las deficiencias cognitivas, a la vez que mantienen la motivación del paciente durante las sesiones de tratamiento.

\subsubsection{Juegos tradicionales en grupo y sus ventajas cognitivas}

A continuación, se nombrarán algunos juegos tradicionales y populares que fomentan el contacto social y evitan el deterioro cognitivo en personas mayores, especialmente trabajando la memoria. Dado que esto es justo lo que se busca lograr con este proyecto, estas actividades servirán como fuente de inspiración para la conceptualización y desarrollo del mismo \parencite{juegosMem2}.

\begin{itemize}[leftmargin=1.5cm, topsep=2.2pt, itemsep=0.5pt]
    \item \textbf{Cada oveja con su pareja}: 
    consiste en agrupar objetos en función de su categoría, utilizando todo tipo de elementos. Desde las cartas de una baraja para agrupar por palos, hasta objetos aleatorios dispuestos, como frutas, verduras... Contribuye a mantener la capacidad intelectual, la memoria y la coordinación visual y manual.
    \item \textbf{Veo-veo}:
    juego mítico que hace las delicias de los más pequeños, y suele ser cuando juegan con alguien más mayor. Uno de los jugadores tendrá que adivinar el objeto elegido por su inicial; se pueden dar pistas sobre el lugar de la sala en que se encuentra.
    \item \textbf{Palabras encadenadas}:
    se trata de coger la última sílaba de una palabra y encadenar con la siguiente; que la última sílaba de esa palabra sea la primera de la siguiente. Fomenta la memoria y la comunicación, además de la atención y la concentración.
    \item \textbf{Puzzle de refranes}: 
    los adultos mayores son grandes conocedores de refranes populares. El juego consistirá en escribir los refranes en dos partes, en trozos de papel separados. Los jugadores deberán enlazar cada parte para completar el refrán, que puede ir orientado a una temática y luego formar un mural.
    \item \textbf{Quién es quién}:
    cada persona es asignada una palabra que todos menos él conocen. Dicha persona tendrá que ir haciendo preguntas de sí o no hasta adivinar quién o qué es.
    \item \textbf{Adivina la canción}:
    se pueden tomar como referencia las canciones más escuchadas del tiempo en el que las personas mayores eran jóvenes. Se reproducirá la canción durante un corto espacio de tiempo y los participantes anotarán el nombre y artista de la canción. Agudeza auditiva, rapidez y capacidad de atención son algunos de sus beneficios.
\end{itemize}

También destacan juegos de mesa como el Scrabble (creatividad y habilidades lingüísticas), Dominó (planificación y estrategia), Bingo (atención y memoria a corto plazo), Pictionary (comunicación no verbal), rompecabezas (concentración y paciencia), etc \parencite{juegos-mesa}.

\subsection{Juegos digitales para adultos mayores}

Una vez explicado el concepto de juego serio y algunos ejemplos concretos, se hará hincapié en un tipo concreto del mismo: los juegos digitales, que deben su nombre a la tecnologías de digitalización en las que están basados.

Los juegos digitales representan una herramienta innovadora y prometedora para abordar los desafíos asociados con el envejecimiento activo y saludable expuestos anteriormente. A través de dispositivos como computadoras, tabletas, consolas de juegos y teléfonos inteligentes, los adultos mayores tienen acceso a una amplia variedad de experiencias de juego que pueden contribuir a su bienestar físico, cognitivo y social.

Muchos de los juegos digitales populares entre adultos mayores están basados en juegos tradicionales, pues la familiaridad de las dinámicas es un factor que contribuye a su popularidad. Ejemplos de aplicaciones:
\begin{itemize}[leftmargin=1.5cm, topsep=2.2pt, itemsep=0.4pt]
    \item \textbf{Sudoku}: el clásico juego de números ayuda a ejercitar la mente y la concentración.
    \item \textbf{Wordscapes}: búsqueda de palabras que desafía la agilidad mental y el vocabulario.
    \item \textbf{Candy Crush Saga}: niveles de rompecabezas que pueden ser adictivos y desafiantes.
    \item \textbf{Jigsaw Puzzles}: selecciones de rompecabezas virtuales de diferentes niveles de dificultad.
    \item \textbf{Solitaire}: versión digital del clásico juego de cartas, fácil de aprender y entretenido.
    \item \textbf{Peak – Brain Games and Training}: variedad de juegos diseñados para ejercitar diferentes áreas del cerebro.
    \item \textbf{\href{https://neuronapp.com.mx/inicio}{Neurona App}}: aplicación que pretende estimular atención, memoria, razonamiento y planificación de adultos mayores mediante diversos juegos.
    \item \textbf{Fit Brain Trainer}: cuenta con 360 juegos de agilidad mental, memoria, capacidad visual y de deducción.
    \item \textbf{NeuroNation}: capaz de personalizar el entrenamiento para cada usuario \parencite{juegosMem1}.
\end{itemize}

Estos juegos son populares entre adultos mayores debido a su accesibilidad, puesto que son apps que pueden ser descargadas fácilmente en dispositivos móviles y tablets; capacidad para ejercitar la mente, ya que la mayoría estimulan partes del cerebro concretas (la memoria, atención y resolución de problemas) y ofrecer entretenimiento sin demasiada complejidad al tener mecánicas de juego bastante simples.

La investigación en este campo se centra en comprender cómo los juegos digitales pueden promover el envejecimiento activo al mantener la mente activa, mejorar la destreza manual y fomentar la interacción social. Además, se busca diseñar juegos que sean accesibles y atractivos para esta población, teniendo en cuenta las necesidades y preferencias específicas de los adultos mayores.

A medida que la tecnología avanza y la aceptación de los juegos digitales entre los adultos mayores aumenta, resulta importante explorar cómo estos juegos pueden desarrollarse e integrarse de manera efectiva para mejorar la calidad de vida de esta población.

\begin{figure}[ht]
    \centering
    \includegraphics[width=0.55\textwidth]{imgs/digital1.jpg}
    \caption{Mejorar de calidad de vida de adultos mayores con las TIC}
    \label{fig:digital1}
\end{figure}

\subsubsection{El proyecto WorthPlay}

En este campo de investigación, destaca, entre otros, el proyecto \textit{WorthPlay} \parencite{WorthPlay2012}, que se enfoca en la investigación y desarrollo de juegos digitales para personas mayores, con el propósito de promover un envejecimiento activo y saludable. En él se reconoce que las tecnologías de la información y la comunicación (TIC) tienen el potencial de reducir el aislamiento social y mejorar el bienestar físico y psicosocial de los adultos mayores.

Para asegurar la efectividad de los juegos digitales, se destaca la necesidad de entender qué aspectos hacen que un juego valga la pena para esta población. Esto implica considerar elementos educativos, de socialización y de entretenimiento, así como la diversidad de preferencias en cuanto a la jugabilidad y el formato del juego.

El proyecto se basa en el campo de la Interacción Persona-Ordenador (IPO), que busca mejorar la experiencia del usuario con las TIC. Se identifican tres olas en la investigación de IPO, desde la ergonomía hasta la experiencia de usuario (UX). WorthPlay se sitúa en la tercera ola, centrándose en la experiencia de juego y la participación del usuario durante un período prolongado en entornos reales.

La investigación también se enfoca en la inclusión social y la superación de barreras de accesibilidad, como la destreza cognitiva y motriz. Se reconocen los estereotipos negativos sobre los adultos mayores en el contexto de los juegos digitales, y se busca desafiar estos estereotipos a través del diseño participativo de juegos.

El proyecto WorthPlay desarrolla prototipos de juegos y los evalúa con la participación activa de personas mayores. Se espera que este enfoque genere resultados significativos tanto para la academia como para la industria de los juegos digitales, contribuyendo a una mejor comprensión de las necesidades y preferencias de los adultos mayores en este campo.


\subsection{Estado del arte de los asistentes virtuales}

En esta sección, se hablará sobre el estado actual de los asistentes virtuales, que se caracteriza por su creciente sofisticación impulsada por la IA, su diversidad en términos de funciones y aplicaciones, y la presencia destacada de plataformas líderes como Alexa, que continúan innovando y expandiendo las fronteras de esta tecnología.

\subsubsection{Influencia de la IA y clasificación}

La inteligencia artificial (IA) es una tecnología que ha evolucionado rápidamente y se ha integrado en la vida cotidiana de las personas, especialmente en la automatización de servicios y la toma de decisiones. Ha encontrado aplicaciones en diversos sectores, como la educación, la salud, la economía y la política, transformando la forma en que se prestan servicios y se interactúa con las organizaciones.

El aumento de la población y el crecimiento exponencial de las ciudades requieren sistemas más rápidos y eficientes para la atención al cliente. A raíz de esta necesidad, se han desarrollado asistentes virtuales inteligentes.

Un asistente virtual es un software con acceso a recursos en línea que emplea técnicas de IA y procesamiento del lenguaje natural para proporcionar soporte en tiempo real a usuarios y otorgar acceso a información relevante sobre los servicios ofrecidos por organizaciones públicas y privadas \parencite{tfgAlexa1}.

La arquitectura de estos sistemas se ha perfeccionado con el tiempo, desde ELIZA, precursor que emulaba a un psicoterapeuta, hasta los chatbots avanzados y versátiles de hoy en día, compatibles con múltiples dispositivos (altavoces, televisores, teléfonos móviles, tablets, etc).

En cuanto a la clasificación, existe más de un criterio para definir de los tipos de asistentes virtuales, pero los principales son: según el grado de interacción con los usuarios, las funciones y finalidades del servicio, los medios de interacción y el grado de afectividad \parencite{asistentesConv}.

\begin{table}[H]
    \centering
    \begin{tabular}{|l|l|}
    \hline
    \textbf{Criterio} & \textbf{Tipos} \\
    \hline
    \multirow{2}{*}{Grado de interacción} & - Dirigidos \\
     & - Conversacionales \\
    \hline
    \multirow{3}{*}{Funciones y finalidades} & - Comunicación y marketing \\
     & - Atención al cliente \\
     & - Mejora de procesos \\
    \hline
    \multirow{3}{*}{Medios de interacción} & - Texto \\
     & - Multimedia \\
     & - Voz \\
    \hline
    \multirow{2}{*}{Grado de afectividad} & - No emocionales \\
     & - Emocionales \\
    \hline
    \end{tabular}
    \caption{Clasificación de asistentes virtuales}
    \label{tab:criterios_asistentes_virtuales}
\end{table}

\begin{itemize}[leftmargin=1.5cm, topsep=2pt, itemsep=1pt]
    \item \textbf{Dirigidos}: realizan preguntas predefinidas a los usuarios mediante elementos fijos, controlando la interacción con el usuario.
    \item \textbf{Conversacionales}: permiten una mayor libertad en las preguntas que el usuario quiere hacer, promoviendo una interacción más natural.
    \item \textbf{Comunicación y marketing}: brindan servicios de consulta dentro de aplicaciones móviles o web.
    \item \textbf{Atención al cliente}: asisten a los usuarios resolviendo sus dudas y consultas a través de conversaciones continuas.
    \item \textbf{Mejora de procesos}: tienen el objetivo de reducir el tiempo dedicado a una área específica.
    \item \textbf{No emocionales}: son los chatbots tradicionales y se limitan a dar respuestas oportunas a las solicitudes del usuario.
    \item \textbf{Emocionales}: diseñados para interactuar y comprender a las personas a través de conversaciones informales, permitiendo una atención personalizada.
\end{itemize}

Con respecto a las tendencias en el desarrollo e implementación de asistentes virtuales en organizaciones públicas y privadas, se observa que actualmente se utilizan asistentes dirigidos, el tipo más dominante es el de atención al cliente, el medio de interacción más empleado es el texto y suelen ser son no emocionales.

\subsubsection{Alexa, el asistente conversacional más extendido hoy en día}

Junto con Siri (Apple) y Google Assistant, Alexa, desarrollada por Amazon, es sin lugar a dudas una de los asistentes virtuales por voz más extendidas. Numerosos estudios confirman el creciente índice de uso e impacto en el mercado que ha supuesto el lanzamiento de Alexa, entre ellos los resultados obtenidos en la clasificación por puntuaciones del \textit{Voice Platform Impact Rating}:

\begin{figure}[ht]
    \centering
    \includegraphics[width=0.6\textwidth]{imgs/grafico-asistentes.jpeg}
    \caption{Ranking de asistentes de voz según el VPIR en 2020 (\href{https://es.statista.com/grafico/22578/clasificacion-de-los-asistentes-de-voz/}{statista})}
    \label{fig:grafico-asistentes}
\end{figure}

Fue lanzado en noviembre de 2014, se encuentra alojado en la nube de Amazon y está disponible en una gran variedad de dispositivos, incluyendo Amazon Echo y Echo Plus, así como también en dispositivos de terceros como smartphones, Raspberry Pi y vehículos.

Actualmente, cuenta con una amplia gama de funciones predefinidas, como manejo de alarmas, notificaciones, calendarios y búsqueda en internet para responder preguntas. Sin embargo, su verdadero potencial radica en las \textit{skills}, similares a aplicaciones de terceros que amplían sus funcionalidades, permitiendo interactuar con distintas compañías y acceder a sus servicios desde un mismo lugar. A nivel mundial, Alexa cuenta con más de 56,000 skills disponibles \parencite{tfgAlexa2}.

Las \textit{skills} de Alexa para el desarrollo de juegos ofrecen una amplia gama de posibilidades para crear experiencias interactivas y entretenidas. Estas habilidades permiten a los desarrolladores crear juegos de diferentes géneros y niveles de complejidad, desde simples juegos de palabras y adivinanzas hasta juegos de aventuras o trivia más elaborados.

Por los motivos mencionados anteriormente, se va a elegir a Alexa como la asistente virtual para el juego a desarrollar.

\subsection{Aplicaciones similares}

\subsubsection{CELIA, mucho más que una asistente}

Existen algunas iniciativas dirigidas a las personas mayores que están en situación de aislamiento como el asistente virtual Celia (\href{https://celiatecuida.com/}{web oficial Celia}), desarrollado por personal del Centro de Investigación en Tecnologías de Telecomunicación de la Universidade de Vigo, atlanTTic, que ya se ha puesto en marcha con éxito ya que su uso es muy sencillo \parencite{celia-app}.

Las personas interesadas pueden acceder a este asistente virtual desde su teléfono móvil a través de la aplicación gratuita de CELIA, y también por WhatsApp, enviando un mensaje de texto o una nota de voz de manera que establecen una conversación con “Celia”. De esta sencilla manera  la persona  mayor que vive sola puede preguntar al asistente virtual al levantarse “¿Qué tiempo va a hacer hoy?” y buscar actividades para acudir como exposiciones, conferencias, conciertos que sean al aire libre o en espacios cubiertos dependiendo de la climatología. 

\begin{figure}[ht]
    \centering
    \includegraphics[width=0.65\textwidth]{imgs/celia.jpg}
    \caption{Celia, una asistente todoterreno 24 horas al día.}
    \label{fig:celia}
\end{figure}

\subsubsection{Juego de los trayectos orientado a la detección del deterioro cognitivo}

Es una aplicación (o skill) basada en Alexa que puede usarse como una herramienta clínica destinada a la evaluación de la capacidad de memoria de trabajo en pacientes. El juego implementado consiste en ir memorizando las calles de Jaén de una ruta seleccionada aleatoriamente del mapa cada vez que se inicia una partida \parencite{tfgAlexa3}.

Es compatible y accesible a través de cualquier dispositivo de la familia de Alexa y viene con un sistema robusto de almacenamiento en la nube, diseñado para almacenar de manera segura y eficiente toda la información de los usuarios y sus interacciones con el sistema.

Todos estos datos pueden ser consultados mediante una app complementaria destinada exclusivamente al uso del especialista. Así, este último podrá evaluar de manera efectiva y eficiente la capacidad de memoria de trabajo de sus pacientes a partir de la información de sus partidas.

\subsubsection{Skill de Alexa para la mejora de la inhibición de respuesta}

Consiste en dos juegos principales (animales y colores) con dos modalidades cada uno, empleando la voz y una interfaz gráfica en un dispositivo de Alexa. Estos están dirigidos principalmente a personas con el trastorno por déficit de atención e hiperactivada (TDAH) o variantes similares \parencite{tfgAlexa1}.

En la primera modalidad del juego de animales, hay 6 rondas y en cada una, se muestra un carrusel de 4 imágenes (2,5 segundos cada una). Se debe pulsar el botón rojo cuando aparezca la imagen que se corresponde al animal que Alexa diga. En el segundo modo de juego, el nombre del animal en cuya aparición hay que presionar el botón viene escrito en pantalla desde el principio, y Alexa dirá el nombre de un animal aleatorio para intentar confundir al jugador.

En cuanto al juego de colores, en la primera variante hay dos pulsadores, un tick y una cruz. Se mostrará por pantalla el nombre de un color, pintado de un color que puede o no ser el mismo. Si coinciden, el jugador debe pulsar el tick y en caso contrario, la cruz. En la segunda versión, se muestran una serie de botones de colores y el jugador debe pulsar la que se corresponda con el color que aparezca escrito.

\begin{figure}[ht]
    \centering
    \includegraphics[width=0.65\textwidth]{imgs/mockupTFG1.JPG}
    \caption{Interfaz gráfica del juego colores v2}
    \label{fig:mockupTFG1}
\end{figure}

Estos cuatro juegos se han desarrollado tras un trabajo de investigación exhaustivo para determinar qué factores proporcionan más beneficios al grupo de personas al que va dirigido.


\newpage
\section{Planificación, metodología y presupuesto}

En esta sección, se van a definir los componentes fundamentales en la gestión del proyecto. Estos permitirán definir claramente los objetivos, los recursos disponibles y cómo se utilizarán para alcanzar los resultados deseados dentro de un marco de tiempo y costos preestablecidos.

\subsection{Planificación temporal}

La planificación establece el marco temporal y los objetivos del proyecto dentro de las fechas límite para su entrega.

\vline

\begin{tabular}{|c|p{8cm}|c|c|c|}
	\hline
	\rowcolor{lightgray}
	\textbf{ID} & \textbf{Tarea} & \textbf{F. inicio} & \textbf{F. final} & \textbf{Duración (d.)} \\
	\hline
	\textbf{1.} & \textbf{Investigación inicial} & \textbf{15 ene.} & \textbf{11 feb.} & \textbf{28} \\
	\hline
	1.1. & Motivación, contexto y objetivos  & 15 ene. & 25 ene. & 11 \\
	\hline
	1.2. & Investigación del estado del arte & 26 ene. & 11 feb. & 17 \\
	\hline
	\textbf{2.} & \textbf{Análisis del problema y elección de temática} & \textbf{12 feb.} & \textbf{21 abr.} & \textbf{70} \\
	\hline
	2.1. & Elección de la temática del juego & 12 feb. & 21 feb. & 10 \\
	\hline
	2.2. & Creación de skill, modelo de interacción y oca básica & 22 feb. & 21 abr. & 59 \\
	\hline
	2.3. & Historias de usuario, requisitos del sistema... & 1 abr. & 21 abr. & 21 \\
	\hline
	\textbf{3.} & \textbf{Diseño} & \textbf{22 abr.} & \textbf{5 may.} & \textbf{14} \\
	\hline
	3.1. & Desarrollo de modelos: arquitectura, conceptual, E/R...  & 22 abr. & 28 abr. & 7 \\
	\hline
	3.2. & Diseño de la interfaz, cuestiones estéticas y usabilidad & 29 abr. & 5 may. & 7 \\
	\hline
	\textbf{4.} & \textbf{Análisis tecnológico y desarrollo de código} & \textbf{6 may.} & \textbf{25 ago.} & \textbf{112} \\
	\hline
	4.1. & Funcionalidad completa del juego (con minijuegos y APL) & 6 may. & 18 ago. & 106 \\
	\hline
	4.2. & Configuración de servicios de AWS: Amazon S3, IAM y DynamoDB  & 7 jul. & 18 ago. & 43 \\
	\hline
	4.3. & Pruebas de ejecución y documentación del código & 19 ago. & 25 ago. & 7 \\
	\hline
	\textbf{5.} & \textbf{Memoria del proyecto} & \textbf{1 feb.} & \textbf{28 ago.} & \textbf{210} \\
	\hline
	\textbf{6.} & \textbf{Últimas pruebas y producto final} & \textbf{26 ago.} & \textbf{1 sep.} & \textbf{7} \\
	\hline
	\rowcolor{lightgray}
	\textbf{} & \textbf{Conjunto total del proyecto} & \textbf{15 ene.} & \textbf{1 sep.} & \textbf{230} \\
	\hline
\end{tabular}


\subsection{Metodología}

La metodología describe las técnicas y procesos que se seguirán para desarrollar el proyecto, incluyendo las herramientas y tecnologías utilizadas, los roles y responsabilidades del equipo y los métodos de comunicación y gestión del proyecto. En este caso, se va a emplear la metodología de \textit{desarrollo ágil}.

El método de desarrollo ágil utiliza un enfoque iterativo y flexible, facilitando la adaptación a cambios y la entrega continua de valor al usuario. Esta metodología es especialmente efectiva en entornos donde las necesidades del usuario y el mercado pueden cambiar rápidamente, y donde la colaboración y la comunicación efectiva entre los miembros del equipo y con el cliente son fundamentales para el éxito del proyecto \parencite{metodologiaAgil}.

El desarrollo ágil para una aplicación se divide en varias etapas que se alinean con las prácticas y principios de dicha metodología. Estas fases incluyen:

\begin{enumerate}
    \item \textbf{Análisis del problema}: se centra en comprender las necesidades del usuario y los requisitos del sistema. En el contexto de una aplicación, implica la creación de historias de usuario y casos de uso, que son esenciales para definir las funcionalidades que la aplicación debe ofrecer.
    \item \textbf{Diseño}: se planifican las soluciones técnicas y se definen los detalles de diseño, como la arquitectura de la aplicación, el modelo E/R y el diseño de la interfaz de usuario. El diseño conceptual y los bocetos y mockups de esta fase preparan el camino para la implementación.
    \item \textbf{Desarrollo}: se trabaja para implementar las soluciones diseñadas. Esto incluye la codificación, integración de componentes y configuración del entorno de desarrollo.
    \item \textbf{Pruebas}: su realización permite corregir errores a tiempo. Esto asegura que la aplicación funcione como se espera y cumpla con los requisitos definidos en las etapas anteriores.
    \item \textbf{Despliegue}: una vez que la aplicación ha sido probada y se ha asegurado de que cumple con los requisitos y expectativas, se despliega para su uso.
    \item \textbf{Revisión y mejora continua}: tras el despliegue, se puede incluir la recopilación de comentarios de los usuarios, la identificación de áreas de mejora en el proceso de desarrollo y las implementaciones de cambios para mejorarla en iteraciones futuras.
\end{enumerate}


\subsection{Presupuesto}

El presupuesto es una herramienta clave para gestionar los recursos del proyecto, desde el asignamiento de recursos humanos hasta la adquisición de equipamiento y materiales necesarios. Permite estimar los costos asociados a cada entregable y recurso requerido, estableciendo un cronograma de gastos y asignando responsabilidades para el control de los mismos.

\subsubsection{Recursos humanos}
Incluye las personas involucradas en el proyecto, tanto en las etapas previas al desarrollo como durante el mismo.

\begin{table}[H]
    \centering
    \begin{tabular}{|c|c|}
    \hline
    \rowcolor{lightgray}
    \textbf{Descripción} & \textbf{Coste (€)}\\
    \hline
    Programador(a) & X \\
    \hline
    Psicólogo/a & X \\
    \hline
    \end{tabular}
    \caption{Presupuesto para recursos humanos}
    \label{tab:presupuesto-personal}
\end{table}

\subsubsection{Hardware}
Todos los dispositivos físicos y electrónicos necesarios para llevar a cabo la aplicación.

\begin{table}[H]
    \centering
    \begin{tabular}{|c|c|}
    \hline
    \rowcolor{lightgray}
    \textbf{Descripción} & \textbf{Coste (€)}\\
    \hline
    HP 15s Intel Core i5-1035G1/16GB/1TB/15.6" & 550-650 \\
    \hline
    Alexa Echo Show & 70 \\
    \hline
    \end{tabular}
    \caption{Presupuesto para hardware}
    \label{tab:presupuesto-hw}
\end{table}

\subsubsection{Software}
Los programas 

\begin{table}[H]
    \centering
    \begin{tabular}{|c|c|}
    \hline
    \rowcolor{lightgray}
    \textbf{Descripción} & \textbf{Coste (€)}\\
    \hline
    Visual Paradigm & 0 \\
    \hline
    Amazon Web Server (AWS) & X/? \\
    \hline
    Dynamo DB & X/? \\
    \hline
    Amazon S3 & X/? \\
    \hline
    TeXstudio & 0 \\
    \hline
    Visual Studio Code & 0 \\
    \hline
    Alexa Developer Console & 0 \\
    \hline
    \end{tabular}
    \caption{Presupuesto para software}
    \label{tab:presupuesto-sw}
\end{table}

\subsubsection{Resumen de presupuesto}
tfnjtrjrf

\begin{table}[H]
    \centering
    \begin{tabular}{|c|c|}
        \hline
        \rowcolor{lightgray}
        \textbf{Descripción} & \textbf{Coste (€)} \\
        \hline
        Recursos humanos & X \\
        \hline
        Hardware & X \\
        \hline
        Software & X \\
        \hline
    \end{tabular}
    \caption{Tabla general de presupuestos}
    \label{tab:presupuesto-total}
\end{table}


\newpage
\section{Análisis del problema}

\subsection{Elección de la temática: el juego de la oca}

Antes de pasar a la etapa de diseño y desarrollo, se debe elegir el tipo de juego a implementar, teniendo en cuenta factores de viabilidad tanto del ámbito tecnológico como del psicológico (¿es apropiado para el grupo demográfico al que va dirigido?).

Como se ha visto en la sección \textit{2.1.3.}, los juegos tradiciones pueden servir como fuente de inspiración para lo que se busca en este proyecto. Sin embargo, existe una gran cantidad de juegos de mesa populares entre los adultos mayores, como es el caso del parchís, el scrabble, el memorama, el dominó, etc. Entonces, ¿por qué decantarse por el juego de la oca?

El principal motivo es la flexibilidad que permite a la hora de diseñar un juego que cumpla unos objetivos específicos. Un ejemplo de ello es la iniciativa de un Centro de Educación Primaria en Murcia, que implementó un programa innovador para la evaluación de las habilidades motrices del estudiantado. \parencite{experienciaOca}

El proceso fue el siguiente: se diseñaron tres tableros de la oca, que se correspondían con los tres ciclos escolares que participaron en el experimento. Cada uno, además de las casillas especiales del juego original (oca, puente, calavera...), disponía de una serie de actividades físicas para poner a prueba a los participantes, entre las que se encuentran: el lanzamiento y captura de objetos, ejercicios de malabares y equilibrio y cooperación con los compañeros.

\begin{figure}[h]
	\centering
	\includegraphics[width=1\textwidth]{imgs/casillas-oca-primaria.JPG}
	\caption{Diseño de la oca para evaluar las habilidades motoras en Educación Primaria \parencite{experienciaOca}}
	\label{fig:casillas-oca-primaria}
\end{figure}

\begin{figure}[h]
	\centering
	\includegraphics[width=0.45\textwidth]{imgs/oca-primaria.JPG}
	\caption{Tablero completo correspondiente a las casillas de la figura superior \parencite{experienciaOca}}
	\label{fig:oca-primaria}
\end{figure}


Esta medida tuvo gran éxito, pues los seguimientos del progreso de los participantes mostraron una mejoría general en sus destrezas físicas y capacidad de trabajo en equipo, así como una participación activa por parte de los estudiantes de Primaria que no se había registrado hasta el momento.

Por tanto, se ha querido replicar de alguna manera ese enfoque, trasladándolo a un contexto donde el grupo objetivo son los adultos mayores, y teniendo en cuenta la diversidad que puede existir dentro del mismo. 

Partiendo de la estructura del tablero original, hay una mayor probabilidad de que los adultos mayores estén familiarizados con su diseño y las reglas básicas, lo que puede contribuir a una experiencia de juego positiva y fluida.

\begin{figure}[H]
	\centering
	\includegraphics[width=0.7\textwidth]{imgs/oca-tradicional.jpg}
	\caption{Un tablero del juego tradicional de la oca}
	\label{fig:oca-tradicional}
\end{figure}


Sin embargo, al igual que la oca adaptada para estudiantes de Educación Primaria vista anteriormente, no se tratará de una oca tradicional cualquiera. 

Pues aparte de las casillas originales y el objetivo de llegar a la meta, incluirá un sistema de puntos, añadiendo un subobjetivo: el de conseguir la mayor puntuación antes de que termine la partida. Los participantes del juego podrán incrementar su marcador a través de una serie de minijuegos, que serán desencadenados cuando los jugadores caigan en determinadas casillas especiales.

También debe evaluarse la viabilidad de los minijuegos propuestos dentro del juego principal, que es la oca, para lo que se ha consultado con un equipo de psicólogos que han participado en varias experiencias de residencias.

\subsubsection{Lista de minijuegos dentro de la oca}
Dada la complejidad de implementación que conllevan algunos, y para evitar saturar a los participantes con demasiadas mecánicas nuevas, se han elegido los siguientes minijuegos:

\begin{enumerate}
	\item \textbf{Preguntas de Trivial V/F}: pondrán a prueba los conocimientos generales de los participantes. Se abordarán categorías distintas (geografía, historia, ciencia, arte y cultura…) pero siempre habrá dos opciones de respuesta, verdadero o falso.
	\begin{itemize}
		\item Alexa: \textit{¿Oceanía es el continente más pequeño de todos?}
		\item Jugador/a: \textit{Verdadero.}
		\item Alexa: \textit{¡Correcto! Has ganado 'x' puntos.}
		
		Alternativamente...
		\item Jugador/a: \textit{Falso.}
		\item Alexa: \textit{Incorrecto, el continente más pequeño del mundo es Oceanía, con 9.000.000 km² de superficie.}
	\end{itemize}
	
	\item \textbf{Conoce a los participantes}: Alexa hace preguntas de "Sí o No" acerca de los otros jugadores, quienes deberán luego confirmar si las respuestas proporcionadas son correctas o no.
	\begin{itemize}
		\item Alexa: \textit{Jugador Rojo, ¿Jugadora Verde ha viajado fuera de su país alguna vez?}
		\item Jugador Rojo: \textit{Sí / No}
		\item Alexa: \textit{Jugadora Verde, ¿es correcta la respuesta?}
		\item Jugadora Verde: \textit{Sí / No}
	\end{itemize}
	
	\item \textbf{Adivina la fecha}: Alexa hace preguntas sobre fechas emblemáticas en las que sucedió algún hecho importante, o del estilo "¿a qué día estamos?", para poner a prueba la memoria de los participantes. Si el primero en responder falla, se pasará al segundo y así sucesivamente hasta llegar de nuevo al primero.
	\begin{itemize}
		\item Alexa: \textit{J1, ¿en qué año se descubrió América?}
		\item J1: \textit{1520}
		\item Alexa: \textit{Incorrecto, la pregunta rebota a J2. ¿En qué año se descubrió América?}
		\item J2: \textit{1492}
		\item Alexa: \textit{¡Correcto! J2 gana 'x' puntos.}
	\end{itemize}
	
	\item \textbf{Recuerda la última casilla}: Alexa preguntará por la casilla más reciente previa a la última tirada de dado; es decir, la casilla donde cayó en el turno anterior.
	\begin{itemize}
		\item Alexa: \textit{¿Cuál fue la última casilla en la que caíste en el turno anterior?}
		\item Jugador/a: \textit{La casilla del sombrero}
		\item Alexa: \textit{¡Correcto! Has ganado 'x' puntos.}
		
		Alternativamente...
		\item Alexa: \textit{Incorrecto, la casilla en la que caíste en el turno anterior fue la casilla del paraguas.}
	\end{itemize}
	
	\item \textbf{Secuencia de palabras}: Alexa dirá una secuencia de 5 palabras que pueden o no estar relacionadas entre sí, y el jugador irá diciendo una por una las palabras que recuerde en el orden original. Si logra acertar un mínimo de 3 palabras, gana la prueba, y si acierta más de 3, conseguirá puntos de bonificación adicionales.
	\begin{itemize}
		\item Alexa: \textit{La secuencia de palabras es: GATO, MAR, CASA, SOL y BALÓN}
		\item Jugador/a: \textit{GATO}
		\item Alexa: \textit{Correcto, ¿siguiente palabra?}
		\item Jugador/a: \textit{MAR}
		\item Alexa: \textit{Correcto, ¿siguiente palabra?}
		\item Jugador/a: \textit{CASA}
		\item Alexa: \textit{Correcto, ¿siguiente palabra?}
		\item Jugador/a: \textit{BALÓN}
		\item Alexa: \textit{Incorrecto, la siguiente palabra era: SOL. Como has acertado 3 sobre 5, has ganado 'x' puntos, enhorabuena.}
		\item ...
	\end{itemize}
\end{enumerate}


\subsection{Historias de usuario y requisitos}

Las historias de usuario son imprescindibles en el desarrollo ágil de software, pero también resultan valiosas en cualquier otra metodología de desarrollo, ya sea tradicional o no. Pues a través de un formato sencillo, recogen de forma clara las necesidades y expectativas de los usuarios y clientes, facilitando la comunicación entre integrantes del equipo de desarrollo y clientes. Un ambiente colaborativo donde las ideas pueden fluctuar a medida que se avanza aumenta las probabilidades de éxito de cualquier proyecto \parencite{introHU}.

El formato típico de una historia de usuario se basa en los  tres elementos siguientes:

\begin{figure}[H]
	\centering
	\includegraphics[width=0.88\textwidth]{imgs/formatoHU.jpg}
	\caption{Patrón general de una historia de usuario}
	\label{fig:formatoHU}
\end{figure}

Esta estructura facilita empatizar con la persona usuaria, entender qué pretende lograr sin entrar en detalles del "cómo" y el valor que aporta al producto, conocido como beneficio. Así, además de sintetizar en gran medida la información, permite cierta flexibilidad y la adaptación a cambios e integración de nuevas ideas durante el proceso de desarrollo.

Las historias de usuario ofrecen una perspectiva más centrada en los usuarios finales, mientras que los requisitos funcionales se limitan a describir el comportamiento del sistema. Entonces, merece la pena considerar ambos para que el producto final no solo cumpla con las especificaciones técnicas, sino que también otorgue valor real a quienes va dirigido.

\subsubsection{Historias de usuario}

Las siguientes historias de usuario establecen los requisitos de sistema del juego de la oca controlada por Alexa.

Estas contemplan desde la creación y configuración de una partida, hasta la gestión de turnos y control del estado del juego, garantizando asistencia activa durante todo el juego.

\begin{table}[H]
    \centering
    \begin{tabular}{|c|c|}
        \hline
        \multicolumn{2}{|c|}{\textbf{HU01}: Iniciar el juego} \\
        \hline
        \textbf{Como} & jugador/a \\
        \hline
        \textbf{Quiero} & poder iniciar el juego de la oca \\
        \hline
        \textbf{Para} & empezar a jugar \\
        \hline
    \end{tabular}
    \caption{Historia de usuario nº 1}
    \label{tab:HU01}
\end{table}

\begin{table}[H]
	\centering
	\begin{tabular}{|c|p{14cm}|}
		\hline
		\multicolumn{2}{|c|}{\textbf{HU02}: Creación de una partida nueva} \\
		\hline
		\textbf{Como} & jugador/a \\
		\hline
		\textbf{Quiero} & poder iniciar un juego nuevo, pudiendo elegir cuántos jugadores van a participar, y si va ser por equipos o individualmente \\
		\hline
		\textbf{Para} & adaptar el juego a la cantidad y tipo de participantes \\
		\hline
	\end{tabular}
	\caption{Historia de usuario nº 2}
	\label{tab:HU02}
\end{table}

\begin{table}[H]
    \centering
    \begin{tabular}{|c|c|}
        \hline
        \multicolumn{2}{|c|}{\textbf{HU03}: Escuchar las reglas} \\
        \hline
        \textbf{Como} & nuevo jugador/a \\
        \hline
        \textbf{Quiero} & pedirle a Alexa que me explique las reglas del juego \\
        \hline
        \textbf{Para} & entender cómo jugar antes de comenzar \\
        \hline
    \end{tabular}
    \caption{Historia de usuario nº 3}
    \label{tab:HU03}
\end{table}

\begin{table}[H]
    \centering
    \begin{tabular}{|c|c|}
        \hline
        \multicolumn{2}{|c|}{\textbf{HU04}: Jugar un turno} \\
        \hline
        \textbf{Como} & jugador/a \\
        \hline
        \textbf{Quiero} & que Alexa tire los dados por mí y mueva mi ficha \\
        \hline
        \textbf{Para} & jugar mi turno \\
        \hline
    \end{tabular}
    \caption{Historia de usuario nº 4}
    \label{tab:HU04}
\end{table}

\begin{table}[H]
    \centering
    \begin{tabular}{|c|c|}
        \hline
        \multicolumn{2}{|c|}{\textbf{HU05}: Guardar/reanudar la partida} \\
        \hline
        \textbf{Como} & usuario/a \\
        \hline
        \textbf{Quiero} & que Alexa tenga la capacidad de guardar y cargar el progreso de la partida actual \\
        \hline
        \textbf{Para} & poder retomar el juego más tarde sin perder el avance \\
        \hline
    \end{tabular}
    \caption{Historia de usuario nº 5}
    \label{tab:HU05}
\end{table}

\begin{table}[H]
    \centering
    \begin{tabular}{|c|c|}
        \hline
        \multicolumn{2}{|c|}{\textbf{HU06}: Finalizar el juego} \\
        \hline
        \textbf{Como} & usuario/a \\
        \hline
        \textbf{Quiero} & poder finalizar el juego en cualquier momento \\
        \hline
        \textbf{Para} & terminar la partida cuando lo desee \\
        \hline
    \end{tabular}
    \caption{Historia de usuario nº 6}
    \label{tab:HU06}
\end{table}

\begin{table}[H]
    \centering
    \begin{tabular}{|c|c|}
        \hline
        \multicolumn{2}{|c|}{\textbf{HU07}: Recibir ayuda} \\
        \hline
        \textbf{Como} & adulto/a mayor \\
        \hline
        \textbf{Quiero} & que Alexa me ofrezca ayuda activa durante el juego \\
        \hline
        \textbf{Para} & saber qué hacer por si me pierdo en algún momento \\
        \hline
    \end{tabular}
    \caption{Historia de usuario nº 7}
    \label{tab:HU07}
\end{table}

En el desarrollo del juego interactivo para Alexa, se han diseñado una serie de minijuegos que tienen como objetivo poner a prueba los conocimientos y la memoria de los participantes. Estos son los listados en la sección \textit{4.1.1}. 

A continuación, se presentan las historias de usuario para cada uno de estos minijuegos, que abarcan desde trivias de conocimiento general hasta desafíos de memoria y observación.

\begin{table}[H]
	\centering
	\begin{tabular}{|c|c|}
		\hline
		\multicolumn{2}{|c|}{\textbf{HU08}: Minijuego verdadero o falso} \\
		\hline
		\textbf{Como} & jugador/a \\
		\hline
		\textbf{Quiero} & responder preguntas de conocimiento general con opciones de verdadero o falso  \\
		\hline
		\textbf{Para} & poner a prueba mis conocimientos en diversas categorías y ganar puntos \\
		\hline
	\end{tabular}
	\caption{Historia de usuario nº 8}
	\label{tab:HU08}
\end{table}

\begin{table}[H]
	\centering
	\begin{tabular}{|c|c|}
		\hline
		\multicolumn{2}{|c|}{\textbf{HU09}: Minijuego conoce a los participantes} \\
		\hline
		\textbf{Como} & jugador/a \\
		\hline
		\textbf{Quiero} & responder preguntas sobre otros jugadores con respuestas de sí o no  \\
		\hline
		\textbf{Para} & para demostrar cuánto conozco a mis compañeros y ganar puntos \\
		\hline
	\end{tabular}
	\caption{Historia de usuario nº 9}
	\label{tab:HU09}
\end{table}

\begin{table}[H]
	\centering
	\begin{tabular}{|c|c|}
		\hline
		\multicolumn{2}{|c|}{\textbf{HU10}: Minijuego adivina la fecha} \\
		\hline
		\textbf{Como} & jugador/a \\
		\hline
		\textbf{Quiero} & responder preguntas sobre fechas emblemáticas o la fecha actual\\
		\hline
		\textbf{Para} & poner a prueba mi memoria histórica y de la actualidad y ganar puntos \\
		\hline
	\end{tabular}
	\caption{Historia de usuario nº 10}
	\label{tab:HU10}
\end{table}

\begin{table}[H]
	\centering
	\begin{tabular}{|c|c|}
		\hline
		\multicolumn{2}{|c|}{\textbf{HU11}: Minijuego recuerda la última casilla} \\
		\hline
		\textbf{Como} & jugador/a \\
		\hline
		\textbf{Quiero} & que Alexa me pregunte por la casilla en la que caí en el turno anterior \\
		\hline
		\textbf{Para} & poner a prueba mi memoria y ganar puntos \\
		\hline
	\end{tabular}
	\caption{Historia de usuario nº 11}
	\label{tab:HU11}
\end{table}

\begin{table}[H]
	\centering
	\begin{tabular}{|c|c|}
		\hline
		\multicolumn{2}{|c|}{\textbf{HU12}: Minijuego secuencia de palabras} \\
		\hline
		\textbf{Como} & jugador/a \\
		\hline
		\textbf{Quiero} & recordar y repetir una secuencia de palabras en el orden original \\
		\hline
		\textbf{Para} & poner a prueba mi memoria y ganar puntos \\
		\hline
	\end{tabular}
	\caption{Historia de usuario nº 12}
	\label{tab:HU12}
\end{table}

\subsubsection{Requisitos funcionales}

Esta sección define y describe las características de alto nivel (requisitos
funcionales) del sistema que son necesarias para cubrir las necesidades de los
usuarios. Se pueden estructurar de la siguiente manera:
\vspace{0.3cm}

\textbf{RF1: Iniciar la skill de Alexa}

Se debe poder lanzar la skill mediante el comando de invocación.
\vspace{0.5cm}

\textbf{RF2: Registro de datos de los participantes}
\begin{itemize}
	\item \textbf{RF2.1}: La skill debe preguntar el modo de juego: por equipos o jugadores individuales. 
	\item \textbf{RF2.2}: La skill debe preguntar el número de jugadores/equipos que van a participar.
	\item \textbf{RF2.3}: La skill debe registrar el nombre de los jugadores o equipos antes de empezar la partida.
\end{itemize}

\textbf{RF3: Simulación de un turno}
\begin{itemize}
    \item \textbf{RF3.1}: La skill debe incluir una función que simule el lanzamiento de dados y determine el número de casillas a avanzar.
    \item \textbf{RF3.2}: La skill debe incluir una función que mueva la ficha del jugador y muestre la casilla en la que ha caído.
\end{itemize}

\textbf{RF4: Gestión del estado del juego}
\begin{itemize}
    \item \textbf{RF4.1}: La skill debe ser capaz de guardar el estado actual del juego.
    \item \textbf{RF4.2}: Relacionado con el anterior, se debe poder reanudar la partida con el estado con el que se ha guardado.
    \item \textbf{RF4.3}: Poder finalizar la partida en cualquier momento (borrando los datos del juego actual).
    \item \textbf{RF4.4}: Poder iniciar una nueva partida en cualquier momento (borrando los datos de la actual).
\end{itemize}

\textbf{RF5: Explicación del juego}
\begin{itemize}
    \item \textbf{RF5.1}: Debe existir una opción dentro de la skill para explicar las reglas y objetivos del juego mediante un comando de voz.
    \item \textbf{RF5.2}: Debe existir una opción dentro de la skill para explicar los tipos de casillas del tablero mediante un comando de voz.
    \item \textbf{RF5.3}: Debe existir una opción dentro de la skill para explicar detalladamente los tipos de minijuegos mediante un comando de voz.
    \item \textbf{RF5.4}: Debe existir una opción dentro de la skill para que Alexa nombre y explique brevemente todos los comandos disponibles.
\end{itemize}

\textbf{RF6: Interacción continua}
\begin{itemize}
	\item \textbf{RF6.1}: La skill debe ser capaz de mantener una interacción continua con el usuario.
    \item \textbf{RF6.2}: La skill debe ofrecer asistencia en todo momento, en casos de que los participantes no sepan qué hacer a continuación o pase cierto tiempo sin recibir una respuesta.
\end{itemize}

\textbf{RF7: Participar en un minijuego}
\begin{itemize}
	\item \textbf{RF7.1}: Cuando se cae en una casilla de minijuego, Alexa debe sacar un elemento aleatorio de la batería de preguntas correspondiente a dicho minijuego.
	\item \textbf{RF7.2}: La skill debe poder capturar la respuesta del participante y verificar si es correcta o no.
\end{itemize}


\subsubsection{Requisitos no funcionales}

En este apartado se pueden ver las diferentes cualidades y restricciones del sistema (requisitos no funcionales) que no se relacionan de forma directa con el comportamiento del mismo.
\vspace{0.3cm}

\textbf{RNF1: Usabilidad}
\begin{itemize}
    \item \textbf{RNF1.1}: La skill debe ser fácil de usar y entender, especialmente diseñada para personas mayores.
    \item \textbf{RNF1.2}: Para evitar dudas, que Alexa explique de forma clara y fácil de entender lo que deben hacer los jugadores.
    \item \textbf{RNF1.3}: Alexa debe dejar un margen flexible de tiempo para esperar una respuesta y si no lo hace, repite la pregunta.
\end{itemize}

\textbf{RNF2: Accesibilidad e interfaz}
\begin{itemize}
    \item \textbf{RNF2.1}: La interfaz debe ser simple y limitarse a mostrar los elementos relevantes de la partida para evitar saturar a los jugadores y jugadoras.
    \item \textbf{RNF2.2}: La fuente y disposición de elementos debe estar adaptada para la compresión y comodidad de los y las participantes.
\end{itemize}

\textbf{RNF3: Rendimiento}
\begin{itemize}
    \item \textbf{RNF3.1}: Las respuestas a los comandos de voz deben ser rápidas, idealmente no superando los 5 segundos.
\end{itemize}

\textbf{RNF4: Fiabilidad}
\begin{itemize}
    \item \textbf{RNF4.1}: La skill debe funcionar correctamente en la mayoría de las interacciones, minimizando errores y malentendidos en el reconocimiento de voz.
\end{itemize}

\newpage
\subsection{Casos de uso y sus correspondientes diagramas}
\subsubsection{Casos de uso}
\subsubsection{Matriz de cobertura de requisitos funcionales}
\subsubsection{Diagramas de secuencia}





\newpage
\section{Diseño}

En esta etapa de la metodología ágil, se debe elaborar un diseño que responda a las dificultades que los adultos mayores puedan experimentar probando juegos digitales. Encontrar el punto medio entre entretenimiento y aprendizaje, hacer una correcta integración de los elementos visuales y garantizar la legibilidad son algunos de los aspectos que se tratarán en este apartado.

\subsection{Diseño de la arquitectura}

La arquitectura completa de la skill se construirá sobre varias herramientas de \textit{Amazon Web Services} (AWS) que serán explicadas con mayor detalle en la sección 6. Estas se encuentran englobadas en \textit{AWS Serverless Platform}, una plataforma que permite la creación de skills sin necesidad de disponer de un servidor propio. 

Las tecnologías avanzadas de AWS más relevantes en el desarrollo de skills son: AWS Lambda, Amazon DynamoDB y Amazon S3.

\begin{figure}[H]
	\centering
	\includegraphics[width=0.98\textwidth]{imgs/arquitectura-skill.png}
	\caption{Arquitectura de una skill de Alexa con las tecnologías AWS \parencite{arquitecturaSkill}}
	\label{fig:arquitectura-skill}
\end{figure}


El proceso de creación de la habilidad para Alexa implica varios pasos clave \parencite{arquitecturaSkill}:
\begin{enumerate}
	\item Definir el modelo de interacción por voz; es decir, cómo los usuarios pueden invocar la skill con diversas intenciones mediante comandos de voz.
	\item Diseñar la interfaz, solo en caso de que se vaya a desplegar en dispositivos con pantallas, para mejorar la experiencia de usuario al mostrar información visual complementaria al audio.
	\item Configurar las tablas en DynamoDB: para el almacenamiento persistente de información, usando una base de datos.
	\item Crear un bucket en S3 donde se alojarán las imágenes y vídeos requeridos para el paso 2.
	\item Programar la skill de Alexa, con funciones que permitan gestionar el flujo de conversación entre los usuarios y Alexa.
	\item Escribir la función Lambda, responsable de procesar las entradas del usuario y devolver las respuestas adecuadas, que peuden incluir datos almacenados en DynamoDB y en S3.
	\item Modificar el rol IAM predeterminado para que la función Lambda actualice la información de la base de datos según sea necesario.
	\item Desplegar y probar la skill para asegurarse de que funciona según lo esperado.
\end{enumerate}

Adicionalmente, para gestionar los elementos visuales que serán mostrados en la pantalla del dispositivo de Alexa, se tiene la siguiente arquitectura para el Lenguaje de Presentación de Alexa, comúnmente conocido como APL.

\begin{figure}[H]
	\centering
	\includegraphics[width=0.98\textwidth]{imgs/arquitectura-apl.png}
	\caption{Arquitectura de una APL con los cuatro actores principales (\href{https://developer.amazon.com/en-US/docs/alexa/alexa-presentation-language/apl-bp-understand-apl-architecture.html}{Alexa Developer Documentation})}
\label{fig:arquitectura-apl}
\end{figure}

Los dos agentes esenciales son:
\begin{itemize}
	\item \textbf{La propia skill}: es la que inicia el proceso al enviar primero la plantilla del documento APL al dispositivo.
	\item \textbf{Los dispositivos con pantalla}: algunos dispositivos Alexa, como el Echo Show, son compatibles con este lenguaje de presentación y se encargan de mostrar por pantalla la plantilla APL. 
\end{itemize}

Los anteriores pueden ser complementados por agentes opcionales como: 
\begin{itemize}
	\item \textbf{Los proveedores de datos}: normalmente bases de datos como DynamoDB, almacenan de forma externa a la skill información relevante que puede ser consultada y/o modificada.
	\item \textbf{Los proveedores de contenido}: como Amazon S3, que incluyen archivos multimedia externos, que se almacenan públicamente en la red y son referenciados desde la skill mediante URLs. 
\end{itemize}

\subsection{GDD: Documento de Diseño del Juego}

Como bien se menciona en el libro \textit{Game design workshop: a playcentric approach to creating innovative games} \parencite{fullerton2008game}, el Game Design Document (GDD) es un documento que integra todos los aspectos relevantes de un videojuego, desde la mecánica y la narrativa hasta los elementos visuales y de audio. 

Es útil porque sirve como roadmap (plan estratégico para alcanzar unos determinados objetivos a largo plazo) para todo el proceso de desarrollo, así como una guía para coordinar las tareas de los miembros del equipo a lo largo del camino hacia alcanzar dichos objetivos.

Se ha hecho una adaptación a una plantilla de uso gratuito para elaborar el documento de diseño del juego. 

\begin{figure}[H]
	\centering
	\includegraphics{imgs/GDD-1.jpg}
	\caption{Game Design Document página 1}
	\label{fig:GDD-1}
\end{figure}

\begin{figure}[H]
	\centering
	\includegraphics[width=1\textwidth]{imgs/GDD-2.jpg}
	\caption{Game Design Document página 2}
	\label{fig:GDD-2}
\end{figure}

\subsection{Modelo conceptual del juego de la oca}

Aunque en una skill de Alexa los modelos conceptuales no se ajustan de la misma forma que lo harían en un sistema orientado a objetos tradicional, sí se puede plantear un diagrama de conceptos que encapsule las clases del juego de la oca y cómo se relacionan entre sí.

Se ha ilustrado el diagrama conceptual de forma aislada a la estructura particular de la skill:

\vline
\begin{figure}[H]
	\centering
	\includegraphics[width=1\textwidth]{imgs/DConcep.jpg}
	\caption{Diagrama conceptual del juego de la oca (\href{https://www.visual-paradigm.com/}{\textit{Visual Paradigm}})}
	\label{fig:DConcep-1}
\end{figure}

\subsection{Diseño de la interfaz de usuario}

Aunque no se vaya a mostrar en la pantalla del dispositivo de Alexa, se ha realizado un diseño con Adobe Photoshop del tablero personalizado que utilizará la skill de la oca, lo que ha ayudado en gran medida a visualizarlo para poder implementarlo con mayor facilidad más adelante.

\begin{figure}[H]
	\centering
	\includegraphics[width=1\textwidth]{imgs/tablero-oca.jpg}
	\caption{Tablero personalizado para la skill del juego de la oca}
	\label{fig:tablero-oca}
\end{figure}
 
La distribución de las casillas es la siguiente: 18 especiales tomadas del juego tradicional (oca, puente, laberinto, pozo, cárcel y hotel), 15 de tipo minijuego y 30 normales (no desencadenan ninguna acción especial, salvo la de meta). Por tanto, los porcentajes redondeados de las proporciones de cada una son: 28,57\%, 23,81\% y 47,62\%, respectivamente. 

\subsubsection{Bocetos y mockups}

\begin{figure}[H]
    \centering
    \includegraphics[width=0.6\textwidth]{imgs/boceto-bienvenida.JPG}
    \caption{Boceto de la pantalla de inicio de la skill (\href{https://www.lucidchart.com/pages/es}{Lucidchart})}
    \label{fig:boceto-bienvenida}
\end{figure}

\begin{figure}[H]
    \centering
    \includegraphics[width=0.6\textwidth]{imgs/boceto-casilla.JPG}
    \caption{Boceto de la pantalla de turno de jugador (\href{https://www.lucidchart.com/pages/es}{Lucidchart})}
    \label{fig:boceto-casilla}
\end{figure}

\subsubsection{Diagrama de flujo entre pantallas}


\subsubsection{Cuestiones de estética, usabilidad y accesibilidad}

Para el estudio de las variables relevantes a la ergonomía del juego, se ha seguido la metodología utilizada en un juego digital para adultos mayores llamado \textit{Solitaire Quiz}. Este está inspirado en el \textit{Solitario}, un juego de cartas tradicional, y a diferencia del original, incluye contenidos didácticos en forma de pequeños cuestionarios. \parencite{diseño2017}.

\begin{table}[H]
	\centering
	\begin{tabular}{|c|p{6cm}|}
		\hline
		\rowcolor{lightgray}
		\textbf{Categoría} & \textbf{Variables}\\
		\hline
		\multirow{3}{*}{Diseño del juego} & - Desafío \\
		& - Contenido del aprendizaje \\
		& - Retroalimentación \\
		\hline
		\multirow{4}{*}{Usabilidad} & - Ambiente externo al juego \\
		& - Ambiente interno al juego \\
		& - Elementos visuales \\
		& - Dispositivos \\
		\hline
		\multirow{3}{*}{Legibilidad} & - Texto \\
		& - Imágenes \\
		& - Audio \\
		\hline
	\end{tabular}
	\caption{Dimensiones y variables de la ergonomía de la app}
	\label{tab:usabilidad}
\end{table}

%http://agora.edu.es/servlet/articulo?codigo=7894537


%https://www.edutec.es/revista/index.php/edutec-e/article/view/1021

\newpage
\section{Análisis tecnológico}


\subsection{Fundamentos de una skill}

Qué es una skill

\subsubsection{Términos comunes en el desarrollo de skills}

intents

utterances

handlers 

slots

\subsubsection{Memoria y persistencia de datos}

session attributes y DynamoDB entre sesiones.

\subsection{Alexa Skills Kit (ASK)}

ASK es un conjunto de APIs y herramientas que facilitan la integración de nuevas habilidades en Alexa, permitiendo crear distintas clases de skills, desde personalizadas hasta otras específicas para vídeo, listas y hogar inteligente.

El tipo de skill que mejor se ajusta a los objetivos preestablecidos es la personalizada o \textit{Custom Skill}, pues esta permite más flexibilidad a los desarrolladores, permitiéndoles adaptarla a las necesidades de la aplicación a desarrollar.

\subsection{AWS Serverless Platform}

Para el desarrollo de skills de Alexa, se puede optar o bien por la función Lambda de AWS, o bien por un servicio web distinto. La primera opción pertenece a un conjunto de herramientas de desarrollador y servicios en la nube de alto rendimiento que componen la Plataforma sin Servidor de AWS.

Se va a utilizar \textbf{AWS Lambda} para la creación de este juego digital, aprovechando así las múltiples funcionalidades y material de apoyo para encaminar el proceso de desarrollo. Además, al poder hacer uso de los otros servicios incluidos en esta plataforma (DynamoDB, Amazon S3, etc), se garantiza cierta centralización y absoluta compatibilidad entre ellos.

\subsection{Alexa Presentation Language (APL)}

La parte visual de la skill se gestiona mediante el lenguaje de presentación de Alexa (APL), a través del envío de documentos APL al dispositivo en forma de una directiva que se verá más adelante.

Un documento APL consiste en un fichero JSON que define la estructura y disposición de elementos a mostrar por la pantalla del dispositivo de Alexa.
A continuación se muestra un ejemplo básico de documento APL que imprime por pantalla una cadena de texto:

\begin{figure}[H]
	\centering
	\includegraphics[width=0.6\textwidth]{imgs/apl-example.JPG}
	\caption{Ejemplo de documento APL básico (\href{https://developer.amazon.com/en-US/docs/alexa/alexa-presentation-language/apl-document.html}{Alexa Developer Documentation})}
	\label{fig:apl-ejemplo}
\end{figure}



\newpage
\section{Desarrollo}

Una vez pasada a la fase de desarrollo, el objetivo es transformar el diseño y requisitos concretados previamente en una aplicación funcional. 

Este proceso implica la creación de la skill, el desarrollo de código y la realización de pruebas, pero no sin antes haber cumplido dos prerrequisitos:
\begin{itemize}
	\item Tener una cuenta en \textit{Alexa developer console}, que es de carácter totalmente gratuito y permitirá el alojamiento y despliegue de la skill. 
	\item Tener una cuenta de \textit{Amazon Web Services} (AWS), para lo que se debe ingresar un método de pago. Esto es necesario en caso de que el uso de recursos exceda los límites de la oferta gratuita y Amazon tenga que cobrar una cantidad adicional.
\end{itemize}

\subsection{Creación de la skill y el modelo de interacción}

Hay una gran cantidad de guías disponibles para aprender a crear una skill desde cero, y la propia \href{https://developer.amazon.com/en-US/docs/alexa/hosted-skills/build-a-skill-end-to-end-using-an-alexa-hosted-skill.html}{página oficial de desarrolladores Alexa} ofrece diversos documentos de apoyo, los cuales se han utilizado como material de referencia en esta etapa del proyecto.

\begin{enumerate}
	\item Una vez iniciada la sesión en la consola de desarrolladores de Alexa, se abre el menú de creación de skills.
	\item Se elige la región donde se van a alojar los servicios AWS predeterminados de la nueva skill, en este caso, la opción \textit{eu-west-1}, localizada en Irlanda, que es la más cercana.
	\item Se selecciona un nombre para la skill y el idioma para el que va a estar implementado el modelo.
	\item Para la decisión del modelo, el que mejor se ajusta a las especificaciones de este proyecto es el \textit{Custom model} (o personalizado), que permite una mayor flexibilidad.
	\item Como la idea es que Alexa se encargue de alojar el backend de la skill, se elige una de las dos opciones de \textit{Alexa-hosted}, en este caso la del entorno de ejecución de Node.js v16.x.
	\item Se puede añadir una plantilla o importar una skill alojada en un repositorio de Git. En este caso, se ha optado por trabajar a partir de una plantilla básica, que solo incluye un ejemplo simple de \enquote{\textit{Hello world}} (ver figura 29). Sobre esta base se añadirán todas las funcionalidades del juego.
\end{enumerate}

\begin{figure}[H]
	\centering
	\includegraphics[width=1\textwidth]{imgs/alexa-dev-console-2.jpg}
	\caption{Menú de creación de una skill de Alexa}
	\label{fig:alexa-dev-console-2}
\end{figure}

Una vez creada y finalizada la configuración básica, se abrirá automáticamente el menú de \textit{Build}, desde el que se pueden ajustar varios aspectos de la skill.

Lo primero es declarar el nombre de invocación, que es con el que se llamará a la función de lanzamiento en el momento en el que se abre la skill, y se puede ajustar desde el menú lateral de \textit{Invocations}. Alexa establece una serie de restricciones sobre este parámetro: no puede incluir artículos o preposiciones, debe componerse de mínimo dos palabras, si contiene números estos deben ser escritos de manera completa, no debe incluir mayúsculas ni palabras reservadas como \enquote{Alexa, skill, app,} etc. 

El nombre de invocación elegido es \enquote{probando oca}, por tanto cuando se quiera iniciar la skill, habrá que decir: \enquote{Alexa, abre probando oca}.

Otro elemento desplegable revelante del menú lateral es el modelo de interacción (\textit{Interaction Model}), donde se definen los \textit{intents}. Inicialmente hay cinco predeterminados, de los cuales cuatro son de carácter obligatorio y por tanto no pueden borrarse:

\begin{itemize}
	\item \textbf{AMAZON.CancelIntent}: cancela la acción actual y termina la interacción con la skill.
	\item \textbf{AMAZON.HelpIntent}: proporciona información sobre cómo usar la habilidad.
	\item \textbf{AMAZON.StopIntent}: detiene la acción en curso y finaliza la interacción con la skill.
	\item \textbf{AMAZON.NavigateHomeIntent}: regresa al inicio de o a la pantalla principal.
	\item \textbf{HelloWorldIntent}: ejecuta la acción personalizada predefinida, que consiste en saludar a la persona usuaria. Es la única que puede eliminarse.
\end{itemize}

También se puede acceder y modificar el archivo JSON del modelo de interacción directamente, aunque es aconsejable utilizar en lugar de ello la interfaz de configuración, ya que cada vez que se monta la skill este fichero se actualiza automáticamente.

Otro elemento interesante son los Assets, que permiten la creación de tipos personalizados de slots, la consulta del historial de compilación de la skill y la configuración del \textit{endpoint}. 

Al tratarse de una skill alojada por Alexa, el endpoint se trata de la función Lambda de AWS encargada de la ejecución del código de la skill. Además, por defecto se dispone de tres de ellos, localizados en distintas regiones del mundo: Virginia del Norte, Irlanda y Oregon.

\begin{figure}[H]
	\centering
	\includegraphics[width=1\textwidth]{imgs/alexa-dev-console-1.jpg}
	\caption{Menú principal de montaje de la skill de Alexa}
	\label{fig:alexa-dev-console-1}
\end{figure}

Si se navega al menú de código desde la barra de herramientas, se puede encontrar la siguiente estructura de archivos:

\begin{figure}[H]
	\centering
	\includegraphics[width=1\textwidth]{imgs/alexa-dev-console-3.jpg}
	\caption{Menú del código de la skill de Alexa}
	\label{fig:alexa-dev-console-3}
\end{figure}

El fichero \textit{index.js} es el más importante, ya que actúa como entrypoint o punto de entrada de la skill, donde se exporta la función de Lambda. Es el que gestiona todos los handlers (manejadores de solicitudes), que determinan el flujo de conversación de Alexa gracias a la capacidad de establecer respuestas concretas a cada intent. Los handlers predefinidos son:
\begin{itemize}
	\item \textbf{LaunchRequestHandler}:  se activa cuando se abre la skill sin especificar un comando, respondiendo con un mensaje de bienvenida.
	\item \textbf{HelloWorldIntentHandler}: responde cuando el usuario invoca a \textit{HelloWorldIntent}, que se limita a saludar.
	\item \textbf{HelpIntentHandler}: maneja el \textit{AMAZON.HelpIntent}, puede invocarse cuando se necesita ayuda.
	\item \textbf{CancelAndStopIntentHandler}: gestiona \textit{AMAZON.CancelIntent} y \textit{AMAZON.StopIntent}, que sirve para detener la skill.
	\item \textbf{FallbackIntentHandler}: se activa se dice algo que no coincide con ninguno de los intents definidos.
	\item \textbf{SessionEndedRequestHandler}: se invoca cuando una sesión termina, ya sea por un comando de usuario o error.
	\item \textbf{IntentReflectorHandler}: útil para la depuración de la skill, ya que responde con el nombre del intent con el que fue llamado. 
	\item \textbf{ErrorHandler}: sirve para la captura de errores de cualquier tipo.
\end{itemize}

Otro archivo esencial en cualquier proyecto realizado con Node.js es \textit{package.json}, donde se definen las dependencias y bibliotecas requeridas, además de permitir la configuración de scripts, pruebas e información acerca de la versión, nombre del proyecto, etc.

Por otro lado, los dos ficheros restantes (\textit{local-debugger.js} y \textit{util.js}) no han sido necesarios para el desarrollo del proyecto, pero pueden servir como base para facilitar la depuración local de la skill y definir funciones de utilidad generales.

\subsection{Configuración de servicios de AWS}

\begin{figure}[H]
	\centering
	\includegraphics[width=1\textwidth]{imgs/aws-console-1.jpg}
	\caption{Menú de la consola de desarrolladores de AWS}
	\label{fig:aws-console-1}
\end{figure}


\subsubsection{Identity and Access Management (IAM)}

\begin{figure}[H]
	\centering
	\includegraphics[width=1\textwidth]{imgs/aws-iam-2.jpg}
	\caption{Menú de creación de un rol IAM}
	\label{fig:aws-iam-2}
\end{figure}

Se ha creado un rol de IAM con la finalidad de otorgar permisos temporales de operaciones de DynamoDB a la skill que se está desarrollando.

\begin{figure}[H]
	\centering
	\includegraphics[width=1\textwidth]{imgs/aws-iam-1.png}
	\caption{Rol creado para la skill y sus permisos en AWS}
	\label{fig:aws-iam-1}
\end{figure}

Para que la skill pueda usar los recursos de AWS que será definidos en las secciones siguientes, se necesita modificar el fichero JSON que especifica las relaciones de confianza del rol. Para ello, se añade la siguiente entrada:

\begin{figure}[H]
	\centering
	\includegraphics[width=1\textwidth]{imgs/aws-iam-3.png}
	\caption{Configuración de las relaciones de confianza del rol}
	\label{fig:aws-iam-3}
\end{figure}

El texto censurado es el ARN (Amazon Resource Name) o identificador del rol de ejecución IAM que la función Lambda asume al ejecutarse la skill de Alexa. Este se puede obtener desde la consola de desarrolladores de Alexa, en el menú del código, con la opción \textit{Integrate}, especialmente diseñada para poder vincular la skill a servicios personales de AWS.

\begin{figure}[H]
	\centering
	\includegraphics[width=1\textwidth]{imgs/aws-iam-4.png}
	\caption{El AWS Lambda Execution Role ARN de la skill}
	\label{fig:aws-iam-4}
\end{figure}

\subsubsection{Amazon Simple Storage Service (S3)}

Como su nombre indica, es un servicio ofrecido en AWS que permite almacenar objetos de diversos tipos (textos, binarios, documentos, audios, comprimidos, etc). Sin embargo, dados los requisitos de este juego, solo será necesario guardar lo siguiente: las imágenes asociadas a cada casilla, la de la pantalla de bienvenida y los vídeos con las animaciones de los lanzamientos de dado.

En Amazon S3, se ha adoptado el término de \textit{bucket} como un contenedor para almacenar objetos. Cada bucket tiene un nombre único y puede simular una estructura de directorios, además pueden ser de acceso público o privado, al igual que sus elementos, que pueden configurarse a través de políticas de acceso.

Para este proyecto, se ha creado un bucket de nombre \textit{bucket-oca} para guardar todos los archivos multimedia que necesita la skill.

\begin{figure}[H]
	\centering
	\includegraphics[width=1\textwidth]{imgs/aws-s3-1.jpg}
	\caption{Bucket de Amazon S3 para la skill de la oca}
	\label{fig:aws-s3-1}
\end{figure}

\subsubsection{DynamoDB}

Como se ha mencionado en la sección 6.X, los datos de una skill no persisten de una sesión a otra, por lo que es necesario un mecanismo para guardarlos. Aquí entra el papel de las bases de datos, en particular DynamoDB, también intergado en AWS.

Esta base de datos es de tipo NoSQL y es una alternativa interesante no solo por su total compatibilidad con las herramientas actuales del proyecto debido a que es otro servicio Amazon, sino también por su alta capacidad de adaptación a cualquier volumen de datos. 

Es conocido por optimizar sus tiempos de respuesta y llevar a cabo un escalado automático en función del tráfico, gracias a su esquema NoSQL, cuyos elementos de las tablas se identifican mediante claves primarias únicas. Estas últimas pueden ser de partición o compuestas (unión de una clave de partición con una de ordenamiento).



\subsection{Implementación y estructura del código}



- Alexa Developer Console
+ Config. skill
+ Simulador Alexa

- APL
+ docs
+ envío datasources

- Estructura del código (carpetas, ficheros, funciones, etc)

\subsection{Pruebas}
- hacer tablas
+ nombre prueba
+ descripción
+ respuesta esperada
+ validación
+ correspondencia CU

\subsection{Seguridad}

La seguridad es uno de los requisitos no funcionales más importantes, junto con la usabilidad. Si bien es cierto que en un juego de la oca no se tratan datos de alta importancia que haya que cifrar, es una skill que hace uso de servicios de \textit{Amazon Web Services} (AWS). Estos últimos están vinculados a una cuenta administradora que gestiona los métodos de pago de los servicios cuando estos superan una determinada cuota de uso. Por tanto, para proteger estos servicios y limitar su uso a skills personalizadas concretas, es necesario controlar quién tiene acceso, siguiendo el principio de mínimo privilegio. Esto puede lograrse de forma eficiente mediante los roles de \textit{Identity and Access Management} (IAM).

Estos roles son similares a los usuarios IAM, en el sentido de que permite gestionar el acceso a determinados servicios de AWS de forma centralizada; sin embargo, es más conveniente ya que en lugar de estar asociada a una única entidad, proporciona credenciales a cualquier persona usuaria con duración de una sesión. Además, facilitan el seguimiento de registros (o \textit{logs}) de acciones entre usuarios y servicios.

%https://docs.aws.amazon.com/IAM/latest/UserGuide/id_roles.html
importancia de los roles para la administración de acceso y la seguridad en AWS.

Aunque Amazon ofrece los servicios de DynamoDB y Amazon S3 de manera gratuita para las Alexa-hosted skills, no es sin limitaciones: solo puede crearse una tabla en Dynamo y hay un límite de almacenamiento y operaciones para el bucket de S3. Es por este motivo que se ha optado expandir los recursos de la skill con dichos servicios alojados en una cuenta personal de AWS. 

Para ello, se han seguido los pasos detallados en la doucmentación oficial de Alexa Developer:

https://developer.amazon.com/en-US/docs/alexa/hosted-skills/alexa-hosted-skills-personal-aws.html


\newpage
\section{Manual de uso}
Explicación de instrucciones del juego

El primer paso es invocar y lanzar la skill, para lo que es necesario decir \textbf{\textit{<<Alexa, abre El Juego de la Oca>>}}.

Alexa dará un mensaje de bienvenida a los jugadores. Para más información acerca del juego, se le puede pedir a Alexa que explique con más profundidad ciertos temas, diciendo: \textbf{\textit{<<Explícame [tema de ayuda]>>}}. Los temas de ayuda disponibles son: 
\begin{enumerate}
	\item \textbf{Las reglas del juego}: ...
	\item \textbf{Las casillas}: consultar escenario C.
	\item \textbf{Los minijuegos}: una explicación de cada minijuego que 
	\item \textbf{Los comandos de voz}: comenta brevemente los comandos de voz que el usuario puede utilizar en cualquier momento.
\end{enumerate}

\vspace{1cm}
\textbf{Escenario A: Iniciar una nueva partida.}
Para comenzar una nueva partida, se debe decir \textbf{\textit{<<Iniciar nueva partida>>}}.

Alexa procederá a pedir de forma secuencial los datos necesarios para empezar partida en el siguiente orden:
\begin{enumerate}
	\item El modo de juego: por equipos o jugadores individuales.
	\item El número de participantes (equipos o jugadores) que van a competir.
	\item El nombre de cada uno de los participantes, que se irá preguntado uno por uno.
\end{enumerate}

Una vez registrados todos los datos, Alexa hará un resumen general de la configuración de la partida y pedirá confirmación, tras lo cual se pasará al escenario C.

\vspace{1cm}
\textbf{Escenario B: Reanudar la partida anterior.}

Si se ha guardado anteriormente una partida, al abrir la skill la próxima vez Alexa dará la opción al usuario de reanudar dicha partida, con la instrucción \textbf{\textit{<<Continuar partida>>}}.

Se cargará el estado de la partida desde la base de datos y se podrá continuar el juego a partir del último turno registrado.

\vspace{1cm}
\textbf{Escenario C: Jugar un turno.}
Durante la partida, Alexa anunciará el turno del jugador o equipo actual.
Para tirar el dado, el participante o integrante del equipo debe decir \textbf{\textit{<<Tirar el dado>>}}, lo que llevará a mostrar la animación y el resultado de la tirada.

Acto seguido, debe decir \textbf{\textit{<<Mover ficha>>}} para poder avanzar tantas posiciones como indique el dado.

El juego lo colocará en la nueva casilla, que dependiendo de su tipo, puede implicar el final de dicho turno o una nueva acción:
 \begin{itemize}
 	\item \textbf{Casilla normal}: se narra un evento concreto en función de la casilla en la que haya caído, que puede variar de un jugador a otro, y finaliza el turno.
 	\item \textbf{Casilla de oca}: se mueve a la siguiente casilla del mismo tipo y puede tirar de nuevo el dado. En el caso en el que haya caído en la oca anterior a la casilla de meta, se traslada a esta última y finaliza el juego.
 	\item \textbf{Casilla de puente}: solo existen dos de este tipo, y caer en una provoca el desplazamiento inevitable a la otra, tras lo cual finaliza el turno. 
 	\item \textbf{Casilla de penalización}: a su vez se distinguen cuatro tipos (pozo, laberinto, hotel y cárcel), en los que varía el número de turnos que hay que esperarse antes de poder avanzar de nuevo.
 	\item \textbf{Casilla de minijuego}: para un desglose de los tipos de juegos y respuestas aceptadas, consultar el escenario D.
 	\item \textbf{Casilla de meta}: se da la enhorabuena a quien haya llegado antes a la meta, se hace un recuento de puntos para ver quién ha acumulado la mayor cantidad, y se da por finalizada la partida.
 \end{itemize}

\vspace{0.5cm}
\textbf{Escenario D: Respuesta a un minijuego}

\vspace{1cm}
\textbf{Escenario E: Terminar la partida}


\begin{table}[H]
	\centering
	\begin{tabular}{|c|c|}
		\hline
		\textbf{Comando} & \textbf{Cuándo puede invocarse}\\
		\hline
		Explícame & En cualquier momento \\
		\hline
		Nueva partida & En cualquier momento \\
		\hline
		Reanudar partida & Al iniciar sesión, si hay una partida guardada \\
		\hline
		Guardar partida & En cualquier momento \\
		\hline
		Terminar partida & En cualquier momento \\
		\hline
		Tirar dado & Al inicio del turno de un participante \\
		\hline
		Mover ficha & Después de una tirada de dado \\
		\hline
	\end{tabular}
	\caption{Lista de comandos de voz y ámbito de uso}
	\label{tab:comandos-voz}
\end{table}

\begin{table}[H]
	\centering
	\begin{tabular}{|c|c|}
		\hline
		\textbf{Minijuego} & \textbf{Tipo de respuesta esperada}\\
		\hline
		Trivial V/F & Verdadero o Falso \\
		\hline
		Conoce a los participantes & Sí o No \\
		\hline
		Adivina la fecha & Una cifra (día, año, etc), un día de la semana, un mes o una estación del año \\
		\hline
		Recuerda la última casilla & El nombre de la casilla en la que cayó el participante en el turno anterior \\
		\hline
		Secuencia de palabras & Una palabra de la lista cada vez \\
		\hline
	\end{tabular}
	\caption{Tipos de respuestas del usuario a los minijuegos}
	\label{tab:respuestas-minijuegos}
\end{table}



\newpage
\section{Conclusiones y trabajos futuros}
En conclusión, este TFG que se integra en el proyecto de investigación “Evaluación del uso de robots sociales y sistemas conversacionales en Residencias y Centros de Día para promover el envejecimiento saludable” desarrolla un juego en Alexa cuyo sector de la población al que se dirige son las personas mayores que están en situación de poco contacto social, como contribución al desarrollo de soluciones innovadoras y efectivas que aborden el problema del aislamiento social en este sector de la población.

Para lo cual se plantea un juego digital que además de ofrecer entretenimiento y diversión, promueve la interacción social, el compromiso cognitivo y emocional. Se trata de una nueva versión interactiva del tradicional Juego de la Oca, con minijuegos incorporados que le dotan de originalidad y mucho entretenimiento.

Alexa es el asistente virtual escogido para el juego a desarrollar, debido a que sus skills ofrecen una amplia gama de posibilidades para crear experiencias interactivas y entretenidas. Estas habilidades permiten a los desarrolladores crear juegos de diferentes géneros y niveles de complejidad, desde simples juegos de palabras y adivinanzas hasta juegos de aventuras o trivial más elaborados

\subsection{Consecución de objetivos}

Los objetivos alcanzados han sido de manera resumida y tras analizar la diversidad de intervenciones digitales dirigidas a este grupo demográfico de personas mayores, el diseño del juego para lo que he investigado las mejores prácticas en el diseño de juegos digitales accesibles para personas mayores. Para lo cual me he documentado sobre investigaciones publicadas sobre el impacto del aislamiento social en personas mayores como el artículo “La soledad y el aislamiento social en las personas mayores” (Arruebarrena y Cabaco, 2020), que define el aislamiento social como “una ausencia objetiva de relaciones/contactos sociales y la soledad como la experiencia subjetiva aversiva que se siente al valorar esas relaciones/contactos sociales como insuficiente en cantidad y/o calidad”.

Además, se han considerado las adaptaciones necesarias para abordar posibles limitaciones físicas y cognitivas de las personas a las que va dirigido en el desarrollo técnico del juego, como seleccionar la plataforma y tecnologías más apropiadas para el desarrollo del mismo, y asegurar la compatibilidad con dispositivos comunes utilizados por personas mayores.

Otro de los objetivos conseguidos ha sido integrar funcionalidades de accesibilidad, como ajustes de tamaño de fuente y navegación simplificada.

Este trabajo como contribución al conocimiento se centra en cómo la tecnología puede mejorar la calidad de vida de las personas mayores, particularmente en el contexto del aislamiento social.

Para la implementación del juego he investigado otros similares así como aplicaciones ya que en la actualidad ya se han presentado algunas iniciativas dirigidas a las personas mayores que están en situación de aislamiento como, por ejemplo, el asistente virtual Celia desarrollado por personal del Centro de Investigación en Tecnologías de Telecomunicación de la Universidade de Vigo, atlanTTic.

Otra de las aplicaciones estudiadas ha sido el Skill de Alexa para la mejora de la inhibición de respuesta, que onsiste en dos juegos principales (animales y colores) con dos modalidades cada uno, empleando la voz y una interfaz gráfica en un dispositivo de Alexa.

\subsection{Trabajos futuros}

Como aplicación práctica de “Participación comunitaria" se podrían hacer pruebas del juego en residencias de personas mayores o en centros de día, con una persona que actuará como guía para llevar un seguimiento de la experiencia de los usuarios.
De esta manera a partir de la retroalimentación de los y las participantes, se podrían hacer modificaciones o ampliaciones de aspectos del juego para adaptarlos mejor a las personas usuarias.

Otras de las áreas que se podrían completar sería la capacidad de guardar más de una partida en la base de datos, en lugar de solo la última. Esta circunstancia sería útil si se fueran a llevar a cabo varias pruebas piloto distintas, con grupos diversos y durante más de una sesión. Así, se podría mantener el progreso de varias partidas simultáneamente.

Además, se podrían ampliar aspectos del juego como hacer varios tableros, implementar minijuegos nuevos, etc, siempre pensando en las personas a las que va dirigida.
Se podría proponer al popular programa de CanalSurTV “La tarde con Juan y Medio”, cuyos invitados son personas mayores que hicieran una sección para practicar con este juegos interactivo, de manera que se hiciera popular entre su audiencia.


\newpage
\section{Bibliografía}
\defbibheading{empty}{}
\renewcommand{\bibitemsep}{1em}
\printbibliography[heading=empty]

\end{document}