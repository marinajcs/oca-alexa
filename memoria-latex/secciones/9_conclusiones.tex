\section{Conclusiones y trabajos futuros}
En conclusión, este TFG que se integra en el proyecto de investigación “Evaluación del uso de robots sociales y sistemas conversacionales en Residencias y Centros de Día para promover el envejecimiento saludable” desarrolla un juego en Alexa cuyo sector de la población al que se dirige son las personas mayores que están en situación de poco contacto social, como contribución al desarrollo de soluciones innovadoras y efectivas que aborden el problema del aislamiento social en este sector de la población.

Para lo cual se plantea un juego digital que además de ofrecer entretenimiento y diversión, promueve la interacción social, el compromiso cognitivo y emocional. Se trata de una nueva versión interactiva del tradicional Juego de la Oca, con minijuegos incorporados que le dotan de originalidad y mucho entretenimiento.

Alexa es el asistente virtual escogido para el juego a desarrollar, debido a que sus skills ofrecen una amplia gama de posibilidades para crear experiencias interactivas y entretenidas. Estas habilidades permiten a los desarrolladores crear juegos de diferentes géneros y niveles de complejidad, desde simples juegos de palabras y adivinanzas hasta juegos de aventuras o trivial más elaborados

\subsection{Consecución de objetivos}

Los objetivos alcanzados han sido de manera resumida y tras analizar la diversidad de intervenciones digitales dirigidas a este grupo demográfico de personas mayores, el diseño del juego para lo que he investigado las mejores prácticas en el diseño de juegos digitales accesibles para personas mayores. Para lo cual me he documentado sobre investigaciones publicadas sobre el impacto del aislamiento social en personas mayores como el artículo “La soledad y el aislamiento social en las personas mayores” (Arruebarrena y Cabaco, 2020), que define el aislamiento social como “una ausencia objetiva de relaciones/contactos sociales y la soledad como la experiencia subjetiva aversiva que se siente al valorar esas relaciones/contactos sociales como insuficiente en cantidad y/o calidad”.

Además, se han considerado las adaptaciones necesarias para abordar posibles limitaciones físicas y cognitivas de las personas a las que va dirigido en el desarrollo técnico del juego, como seleccionar la plataforma y tecnologías más apropiadas para el desarrollo del mismo, y asegurar la compatibilidad con dispositivos comunes utilizados por personas mayores.

Otro de los objetivos conseguidos ha sido integrar funcionalidades de accesibilidad, como ajustes de tamaño de fuente y navegación simplificada.

Este trabajo como contribución al conocimiento se centra en cómo la tecnología puede mejorar la calidad de vida de las personas mayores, particularmente en el contexto del aislamiento social.

Para la implementación del juego he investigado otros similares así como aplicaciones ya que en la actualidad ya se han presentado algunas iniciativas dirigidas a las personas mayores que están en situación de aislamiento como, por ejemplo, el asistente virtual Celia desarrollado por personal del Centro de Investigación en Tecnologías de Telecomunicación de la Universidade de Vigo, atlanTTic.

Otra de las aplicaciones estudiadas ha sido el Skill de Alexa para la mejora de la inhibición de respuesta, que onsiste en dos juegos principales (animales y colores) con dos modalidades cada uno, empleando la voz y una interfaz gráfica en un dispositivo de Alexa.

\subsection{Trabajos futuros}

Como aplicación práctica de “Participación comunitaria" se podrían hacer pruebas del juego en residencias de personas mayores o en centros de día, con una persona que actuará como guía para llevar un seguimiento de la experiencia de los usuarios.
De esta manera a partir de la retroalimentación de los y las participantes, se podrían hacer modificaciones o ampliaciones de aspectos del juego para adaptarlos mejor a las personas usuarias.

Otras de las áreas que se podrían completar sería la capacidad de guardar más de una partida en la base de datos, en lugar de solo la última. Esta circunstancia sería útil si se fueran a llevar a cabo varias pruebas piloto distintas, con grupos diversos y durante más de una sesión. Así, se podría mantener el progreso de varias partidas simultáneamente.

Además, se podrían ampliar aspectos del juego como hacer varios tableros, implementar minijuegos nuevos, etc, siempre pensando en las personas a las que va dirigida.
Se podría proponer al popular programa de CanalSurTV “La tarde con Juan y Medio”, cuyos invitados son personas mayores que hicieran una sección para practicar con este juegos interactivo, de manera que se hiciera popular entre su audiencia.
