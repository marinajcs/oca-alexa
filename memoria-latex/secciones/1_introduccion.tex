\section{Introducción}
Este Trabajo de Fin de Grado se enfoca en el desarrollo de un juego para asistentes conversacionales dirigido a personas mayores con mayor riesgo de aislamiento social. Reconociendo los desafíos que enfrenta este grupo demográfico en la era digital, el proyecto busca integrar la tecnología de manera accesible y lúdica. El objetivo es no solo fomentar la conexión social, sino también promover la estimulación cognitiva y el bienestar emocional de las personas mayores.

Dado que las personas mayores están menos familiarizadas con el uso de herramientas digitales en su vida cotidiana puede parecer difícil que se sientan atraídas por juegos que se basan en la tecnología, sin embargo si las personas usuarias en un primer momento comprueban que no solo pueden acceder a este entretenimiento si no que, además se incentivan sus relaciones personales, conocen nuevas amistades y se abre un mundo nuevo de ocio y desarrollo intelectual, comprobaremos que su aislamiento se reduce día a día.

La idea detrás de este TFG podría resumirse con la siguiente cita: \enquote{Las actividades físicas, cognitivas y emocionales en edades avanzadas son cruciales para estimular la actividad cerebral y contribuir al mantenimiento de la calidad de vida. Además de los beneficios físicos y cerebrales, los juegos estimulan la interacción social y contribuyen a la socialización y al mantenimiento de la salud emocional y afectiva} \parencite{intro3}.


\subsection{Motivación y contexto}

Según se menciona en el artículo \textit{La soledad y el aislamiento social en las personas mayores} \parencite{ArruebarrenaCabaco2020}, \enquote{El aislamiento social se define como una ausencia objetiva de relaciones/contactos sociales y la soledad
como la experiencia subjetiva aversiva que se siente al valorar esas relaciones/contactos sociales como
insuficiente en cantidad y/o calidad}

En los últimos tiempos, el tema de la soledad en las personas mayores ha ganado atención en los medios de comunicación, describiéndose como una \enquote{epidemia} en aumento. Aunque no hay evidencia sólida que respalde la idea de una nueva epidemia, las dificultades metodológicas y la falta de consenso en la medición de la soledad limitan la capacidad de confirmar si las personas mayores se sienten más solas que antes.

A pesar de estas limitaciones, estudios indican tasas de soledad entre las personas mayores en España, oscilando entre el 14\% y el 24\%, e incluso alcanzando el 40\% en algunos casos. Uno de los factores que contribuyen a esta percepción es el aumento en el número de personas mayores viviendo solas. Se proyecta que para 2050, aproximadamente un tercio de la población tendrá más de 65 años, lo que por lógica implica un aumento en el número de personas mayores que viven solas.

\begin{figure}[ht]
    \centering
    \includegraphics[width=0.98\textwidth]{imgs/piramide-poblacion.jpg}
    \caption{Pirámides de población de España en futuros años (\href{https://www.geriatricarea.com/2020/09/25/uno-de-cada-tres-espanoles-tendra-65-o-mas-anos-en-el-2050/}{geriatricarea.com})}
    \label{fig:piramide-poblacion}
\end{figure}

El aumento de la esperanza de vida de la población adulta en nuestro país ha supuesto que las personas mayores, muchas de ellas que viven solas y poseen nivel económico y cultural medio, supongan un sector amplio de la población que requiere de nuevas formas de ocio y de relacionarse socialmente. Muchas de estas personas viven solas y el sedentarismo y la falta de interacción social les produce que su desarrollo cognitivo se vea mermado.

También la frecuente automarginación de este grupo demográfico para el uso de herramientas digitales como juegos o aplicaciones, el fenómeno conocido de forma general como brecha digital, contribuye a un mayor aislamiento social. \enquote{Muchos adultos mayores tienen acceso a dispositivos móviles, pero no pueden aprovecharlos completamente debido a la falta de conocimiento o el miedo a salir de su zona de confort. Esto resulta en barreras emocionales, dificultades para adquirir nuevas habilidades tecnológicas} \parencite{intro1}. 

A pesar de lo mencionado en el párrafo anterior, el segmento de edad mayor de 60 años no es ajeno a la realidad de que las formas de socialización del siglo XXI están vinculadas a los avances tecnológicos y a las nuevas experiencias lúdicas. Como se señala en \textit{Las competencias digitales en personas mayores: de amenaza a oportunidad}: \enquote{el potencial que para las personas mayores ofrece el uso habitual de las TIC es enorme, con una larga lista de oportunidades existentes para el beneficio de este colectivo que deben ser aprovechadas} \parencite{intro4}.

Un juego digital que suponga entretenimiento para las personas mayores, al mismo tiempo que una mejora de memoria y ampliación de lenguaje y percepción puede significar un estimulante cambio en su día a día. 

\subsubsection{Impacto social y tecnológico de la pandemia de COVID-19}

Durante la pandemia por el COVID-19 millones de personas en el mundo tuvieron que pasar en pocos días de trabajo presencial al online y en sus relaciones personales debieron adaptar sus costumbres al nuevo escenario virtual.

De esta manera, personas acostumbradas a comunicarse solo por llamadas telefónicas, y muy poco mediante telefonía móvil, se volvieron usuarias de videollamadas diarias ante la necesidad de mantenerse conectados con familiares y amigos, combinada con las restricciones de movimiento.

Este cambio tuvo un impacto profundo en la vida diaria de las personas mayores. Por un lado, les brindó una forma vital de mantenerse conectados con sus seres queridos, incluso cuando no podían reunirse físicamente debido a las medidas de distanciamiento social. Esto ayudó a reducir un poco el riesgo de aislamiento social y proporcionó un medio para el apoyo emocional y la interacción social, lo cual es esencial para su bienestar mental y emocional.

Sin embargo, la transición a las tecnologías digitales también presentó desafíos, especialmente para aquellos menos familiarizados con ellas. Algunos enfrentaron dificultades técnicas al principio, como la configuración de aplicaciones o la resolución de problemas de conexión. Además, la dependencia excesiva de la tecnología para la comunicación también puede aumentar la brecha digital entre aquellos que tienen acceso y conocimientos tecnológicos y aquellos que no los tienen, lo que potencialmente podría aumentar el riesgo de exclusión social para algunos adultos mayores.

A pesar de estos desafíos, la pandemia actuó como un catalizador para la adopción de tecnología entre las personas mayores, lo que les permitió permanecer conectados y participar en la sociedad de manera más activa, incluso en tiempos de crisis. Como resultado, muchas personas mayores han incorporado el uso de tecnologías digitales en su vida diaria incluso después de que levantaran las restricciones de la pandemia, lo que les brinda nuevas oportunidades de participación social y acceso a recursos y servicios en línea \parencite{intro2}.

\subsection{Objetivos}
El presente trabajo tiene como objetivo principal concebir, desarrollar y evaluar un juego digital como contribución al desarrollo de soluciones innovadoras y efectivas que aborden el problema del aislamiento social en las personas mayores.

Este juego no solo buscará proporcionar entretenimiento y diversión, sino que también se centrará en promover la interacción social, el compromiso cognitivo y emocional, y en general, mejorar la calidad de vida de las personas mayores.
Para lograr este objetivo principal, se puede descomponer en varios subobjetivos más específicos que vienen ilustrados en la siguiente tabla:

\newpage
\begin{table}[t]
  \centering
  \begin{tabular}{| c | p{9.6cm} |}
    \hline
    \textbf{Subobjetivo} & \textbf{Descripción} \\
    \hline
    1. Revisión de la literatura & 
        1.1. Identificar las investigaciones clave sobre el impacto del aislamiento social en personas mayores \newline
        \vspace{0.2cm}
        1.2. Analizar la diversidad de intervenciones digitales dirigidas a este grupo demográfico \vspace{0.2cm} \\
    \hline
    2. Diseño del juego &
        2.1. Investigar las mejores prácticas en el diseño de juegos digitales accesibles para personas mayores \newline
        \vspace{0.2cm}
        2.2. Considerar las adaptaciones necesarias para abordar posibles limitaciones físicas y cognitivas
        \vspace{0.2cm} \\
    \hline
    3. Desarrollo técnico & 
        3.1. Seleccionar la plataforma y tecnologías más apropiadas para el desarrollo del juego \newline
        \vspace{0.2cm}
        3.2. Asegurar la compatibilidad con dispositivos comunes utilizados por personas mayores \newline
        \vspace{0.2cm}
        3.3. Integrar funcionalidades de accesibilidad, como ajustes de tamaño de fuente y navegación simplificada \vspace{0.2cm} \\
    \hline
    4. Contribución al conocimiento & 
    4.1. Contribuir al creciente cuerpo de conocimientos sobre cómo la tecnología puede mejorar la calidad de vida de las personas mayores, particularmente en el contexto del aislamiento social. 
    \vspace{0.2cm} \\
    \hline
    5. Participación comunitaria (TBD) & 
        5.1. Colaborar con comunidades de personas mayores para obtener retroalimentación durante el desarrollo del juego. \newline
        \vspace{0.2cm}
        5.2. Organizar pruebas piloto y grupos de enfoque para evaluar la experiencia del usuario. \vspace{0.2cm} \\
    \hline
    \end{tabular}
  \caption{Subobjetivos del trabajo.}
  \label{tab:subobjetivos}
\end{table}